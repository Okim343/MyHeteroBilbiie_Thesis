\documentclass[a4paper,12pt]{article} % Set documentclass article and set font size

%%%%%%%%%%%%%%%%%%%%%%%%%%%%%%%%%%%%%%%%%%%%%%%%%%%%%%%%%%%%%%%%%%%%%%%%%%%%%
%%%%%%%%%%%%%%%%%%%%%%%%%%%%%%%%%%%%%%%%%%%%%%%%%%%%%%%%%%%%%%%%%%%%%%%%%%%%%
%%%%%%%%%%%%%%%%%%%%%%%%%%%%%%%%%%%%%%%%%%%%%%%%%%%%%%%%%%%%%%%%%%%%%%%%%%%%%
% Packages

\usepackage[utf8]{inputenc}
\usepackage[british]{babel}
\usepackage[british]{isodate}
\usepackage{csquotes}
\usepackage{mathtools}
\usepackage{amssymb}
\usepackage{geometry}
\usepackage[
backend=biber,
style=apa,
sorting=nyt, % Sort by name, year, title
backref=true,
dateabbrev=false,
language=british,
urldate=long
]{biblatex}
\usepackage{xcolor}
\usepackage[toc,page]{appendix}
\usepackage{svg}
\usepackage{float}
\usepackage{longtable}
\usepackage{multirow}
\usepackage{multirow,multicol,makecell,booktabs}
\usepackage[justification=centering]{caption}
\usepackage{subcaption}
\usepackage{setspace}
\usepackage{ragged2e}
\usepackage{fancyhdr}
\usepackage{enumitem}
\usepackage{tikz}
\usepackage{rotating}
\usepackage{hyperref} % Load after everything else
\usepackage[symbols, nopostdot, automake, nonumberlist]{glossaries}

%%%%%%%%%%%%%%%%%%%%%%%%%%%%%%%%%%%%%%%%%%%%%%%%%%%%%%%%%%%%%%%%%%%%%%%%%%%%%
%%%%%%%%%%%%%%%%%%%%%%%%%%%%%%%%%%%%%%%%%%%%%%%%%%%%%%%%%%%%%%%%%%%%%%%%%%%%%
%%%%%%%%%%%%%%%%%%%%%%%%%%%%%%%%%%%%%%%%%%%%%%%%%%%%%%%%%%%%%%%%%%%%%%%%%%%%%
% Custom Settings

\addbibresource{master_thesis.bib} % Add bibliography source

\setlength\bibitemsep{1.5\itemsep} % Adjust distance of reference entries

\renewcommand*\appendixpagename{\Large Appendices} % Adjust size of title of the appendices section

\renewcommand*{\finentrypunct}{}
\renewbibmacro*{pageref}{%
  \addperiod% NEW
  \iflistundef{pageref}
    {}
    {\newline\footnotesize\printtext[parens]{%
       \ifnumgreater{\value{pageref}}{1}
         {\bibstring{backrefpages}\ppspace}
     {\bibstring{backrefpage}\ppspace}%
       \printlist[pageref][-\value{listtotal}]{pageref}}}}%
\DefineBibliographyStrings{english}{
  backrefpage={Cited on page},
  backrefpages={Cited on pages}
}

\geometry{ % Adjust paper geometry
 a4paper,
 right=25mm,
 bottom=20mm,
 left=25mm,
 top=25mm}
\urlstyle{same} % Make URL font the same with the rest of the document
\onehalfspacing % Adjust spacing
%\setstretch{1.75}
%\doublespacing
\allowdisplaybreaks % Allow for breaks in equations across pages
\addto\captionsbritish{ % Replace "english" with the language you use
  \renewcommand{\contentsname}%
    {Table of Contents}%
} % Adjust title of the table of contents
\numberwithin{equation}{section} % Number equations with sections
\numberwithin{figure}{section}
\numberwithin{table}{section}
%\setcounter{tocdepth}{2} % Set maximum level to which sections are displayed in the table of contents

\newcommand{\appendixsubsection}[1]{
    \stepcounter{subsection}
    \subsection*{\Alph{section}.\arabic{subsection}\hspace{1em}{#1}}
}


\pagestyle{plain}

% Define a custom name format that prints names of authors in references in bold
\DeclareNameWrapperFormat{sortname}{\mkbibbold{#1}}
\DeclareNameWrapperAlias{author}{sortname}

% Add definition format
\newtheorem{definition}{Definition}

% Make links of glossary entries black
\makeatletter
\newcommand*{\glsplainhyperlink}[2]{%
    \begingroup%
      \hypersetup{hidelinks}%
      \hyperlink{#1}{#2}%
    \endgroup%
}
\let\@glslink\glsplainhyperlink
\makeatother

%%%%%%%%%%%%%%%%%%%%%%%%%%%%%%%%%%%%%%%%%%%%%%%%%%%%%%%%%%%%%%%%%%%%%%%%%%%%%
%%%%%%%%%%%%%%%%%%%%%%%%%%%%%%%%%%%%%%%%%%%%%%%%%%%%%%%%%%%%%%%%%%%%%%%%%%%%%
%%%%%%%%%%%%%%%%%%%%%%%%%%%%%%%%%%%%%%%%%%%%%%%%%%%%%%%%%%%%%%%%%%%%%%%%%%%%%
% Glossary
\makeglossaries 

% Glossary entries
\newglossaryentry{nk}{
name=NK,
description={New Keynesian}
}

\newglossaryentry{ha}{
name=HA,
description={Heterogeneous Agent(s)}
}

\newglossaryentry{hank}{
name=HANK,
description={Heterogeneous Agent New Keynesian}
}

\newglossaryentry{gfc}{
name=GFC,
description={Great Financial Crisis}
}

\newglossaryentry{ltv}{
name=LTV,
description={Loan-to-Value}
}

\newglossaryentry{dsge}{
name=DSGE,
description={Dynamic Stochastic General Equilibrium}
}

\newglossaryentry{egm}{
name=EGM,
description={Endogenous Gridpoints Method}
}

\newglossaryentry{iou}{
name=IOU,
description={I Owe You}
}

\newglossaryentry{ghh}{
name=GHH,
description={Greenwood-Hercowitz-Huffman (\cite{ghh1988})}
}

\newglossaryentry{ar}{
name=AR,
description={Autoregressive}
}

\newglossaryentry{zlb}{
name=ZLB,
description={Zero Lower Bound}
}

\newglossaryentry{mpc}{
name=MPC,
description={Marginal Propensity to Consume}
}

\newglossaryentry{fof}{
name=FoF,
description={Flow of Funds}
}

\newglossaryentry{nkpc}{
name=NKPC,
description={New Keynesian Phillips Curve}
}

\newglossaryentry{foc}{
name=FOC,
description={First Order Condition}
}

\newglossaryentry{irf}{
name=IRF,
description={Impulse Response Function}
}

\newglossaryentry{crra}{
name=CRRA,
description={Constant Relative Risk Aversion}
}

\newglossaryentry{gdp}{
name=GDP,
description={Gross Domestic Product}
}

%%%%%%%%%%%%%%%%%%%%%%%%%%%%%%%%%%%%%%%%%%%%%%%%%%%%%%%%%%%%%%%%%%%%%%%%%%%%%
%%%%%%%%%%%%%%%%%%%%%%%%%%%%%%%%%%%%%%%%%%%%%%%%%%%%%%%%%%%%%%%%%%%%%%%%%%%%%
%%%%%%%%%%%%%%%%%%%%%%%%%%%%%%%%%%%%%%%%%%%%%%%%%%%%%%%%%%%%%%%%%%%%%%%%%%%%%
% hyperref package
\usepackage{footnotebackref} % Brings you back to where the footnote is in the text
\hypersetup{ 
    colorlinks,
    linkcolor=blue,
    filecolor=blue,  
    citecolor=blue,
    urlcolor=blue} % Setup with the hyperref package

%%%%%%%%%%%%%%%%%%%%%%%%%%%%%%%%%%%%%%%%%%%%%%%%%%%%%%%%%%%%%%%%%%%%%%%%%%%%%
%%%%%%%%%%%%%%%%%%%%%%%%%%%%%%%%%%%%%%%%%%%%%%%%%%%%%%%%%%%%%%%%%%%%%%%%%%%%%
%%%%%%%%%%%%%%%%%%%%%%%%%%%%%%%%%%%%%%%%%%%%%%%%%%%%%%%%%%%%%%%%%%%%%%%%%%%%%
% Formalities
\title{Technology Choice and Firm Composition in an RBC Model with Endogenous Entry}
\author{}
\date{}

%%%%%%%%%%%%%%%%%%%%%%%%%%%%%%%%%%%%%%%%%%%%%%%%%%%%%%%%%%%%%%%%%%%%%%%%%%%%%
%%%%%%%%%%%%%%%%%%%%%%%%%%%%%%%%%%%%%%%%%%%%%%%%%%%%%%%%%%%%%%%%%%%%%%%%%%%%%
%%%%%%%%%%%%%%%%%%%%%%%%%%%%%%%%%%%%%%%%%%%%%%%%%%%%%%%%%%%%%%%%%%%%%%%%%%%%%
%%%%%%%%%%%%%%%%%%%%%%%%%%%%%%%%%%%%%%%%%%%%%%%%%%%%%%%%%%%%%%%%%%%%%%%%%%%%%
%%%%%%%%%%%%%%%%%%%%%%%%%%%%%%%%%%%%%%%%%%%%%%%%%%%%%%%%%%%%%%%%%%%%%%%%%%%%%
%%%%%%%%%%%%%%%%%%%%%%%%%%%%%%%%%%%%%%%%%%%%%%%%%%%%%%%%%%%%%%%%%%%%%%%%%%%%%
% Begin Document
\begin{document}
\selectlanguage{british}
\pagenumbering{gobble} % Turn page numbering off
\maketitle % Create title

\vspace{4cm}
\begin{center}
Master Thesis Presented to the\\
Department of Economics at the\\
Rheinische Friedrich-Wilhelms-Universität Bonn\\
\vspace{1cm}
In Partial Fulfillment of the Requirements for the Degree of\\
Master of Science (M.Sc.)
\end{center}

\vspace{5cm}
\begin{center}
\begin{tabular}{ l l }
Supervisors: & Prof.~Dr.~Christian Bayer \\
& Dr.~Zheng Gong \\
& \\
& \\
Date of Submission: & \printdate{2025-7-1}\\
Author: & Enrico Truzzi\\
Matriculation Number: & 50182495
\end{tabular}
\end{center}

%%%%%%%%%%%%%%%%%%%%%%%%%%%%%%%%%%%%%%%%%%%%%%%%%%%%%%%%%%%%%%%%%%%%%%%%%%%%%
%%%%%%%%%%%%%%%%%%%%%%%%%%%%%%%%%%%%%%%%%%%%%%%%%%%%%%%%%%%%%%%%%%%%%%%%%%%%%
%%%%%%%%%%%%%%%%%%%%%%%%%%%%%%%%%%%%%%%%%%%%%%%%%%%%%%%%%%%%%%%%%%%%%%%%%%%%%
\newpage
\pagenumbering{roman} % Turn page numbering to small roman
\setcounter{tocdepth}{2}
{ \hypersetup{hidelinks} \tableofcontents } % Hide links in table of contents

\newpage
{ \hypersetup{hidelinks} \listoffigures } % Hide links in list of figures

\newpage
{ \hypersetup{hidelinks} \listoftables } % Hide links in list of tables

% Add here list of acronyms
\newpage
\setlist[description]{leftmargin=!, labelwidth=5em} % Change for glossaries
\printglossary[title=List of Acronyms]
\setlist[description]{style=standard} % Reset settings back to default


%%%%%%%%%%%%%%%%%%%%%%%%%%%%%%%%%%%%%%%%%%%%%%%%%%%%%%%%%%%%%%%%%%%%%%%%%%%%%
%%%%%%%%%%%%%%%%%%%%%%%%%%%%%%%%%%%%%%%%%%%%%%%%%%%%%%%%%%%%%%%%%%%%%%%%%%%%%
%%%%%%%%%%%%%%%%%%%%%%%%%%%%%%%%%%%%%%%%%%%%%%%%%%%%%%%%%%%%%%%%%%%%%%%%%%%%%
\newpage
\begin{abstract} % Create abstract

.

\begin{center}
\rule{10cm}{0.4pt}
\end{center}

\noindent
\textit{JEL Classification:} E12, E21, E32 % JEL topic classification of paper

\noindent
\textit{Keywords:} Firm Heterogeneity, RBC Model % Keywords of paper
\end{abstract}

%%%%%%%%%%%%%%%%%%%%%%%%%%%%%%%%%%%%%%%%%%%%%%%%%%%%%%%%%%%%%%%%%%%%%%%%%%%%%
%%%%%%%%%%%%%%%%%%%%%%%%%%%%%%%%%%%%%%%%%%%%%%%%%%%%%%%%%%%%%%%%%%%%%%%%%%%%%
%%%%%%%%%%%%%%%%%%%%%%%%%%%%%%%%%%%%%%%%%%%%%%%%%%%%%%%%%%%%%%%%%%%%%%%%%%%%%
\newpage
\pagenumbering{arabic} % Turn page numbering to small arabic
\pagestyle{fancy}
\fancyhf{}
\fancyhead[C]{\leftmark}
\fancyfoot[C]{\thepage}

%%%%%%%%%%%%%%%%%%%%%%%%%%%%%%%%%%%%%%%%%%%%%%%%%%%%%%%%%%%%%%%%%%%%%%%%%%%%%
%%%%%%%%%%%%%%%%%%%%%%%%%%%%%%%%%%%%%%%%%%%%%%%%%%%%%%%%%%%%%%%%%%%%%%%%%%%%%
%%%%%%%%%%%%%%%%%%%%%%%%%%%%%%%%%%%%%%%%%%%%%%%%%%%%%%%%%%%%%%%%%%%%%%%%%%%%%
\thispagestyle{plain}
\section{Introduction}
\label{sec:introduction}


%%%%%%%%%%%%%%%%%%%%%%%%%%%%%%%%%%%%%%%%%%%%%%%%%%%%%%%%%%%%%%%%%%%%%%%%%%%%%
%%%%%%%%%%%%%%%%%%%%%%%%%%%%%%%%%%%%%%%%%%%%%%%%%%%%%%%%%%%%%%%%%%%%%%%%%%%%%
%%%%%%%%%%%%%%%%%%%%%%%%%%%%%%%%%%%%%%%%%%%%%%%%%%%%%%%%%%%%%%%%%%%%%%%%%%%%%
\section{Related Literature}
\label{sec:literature}



I first start this review of the literature by addressing \textcite{bilbiie2012endogenous} (henceforth BGM), since it represents the core framework behind the model developed in this thesis. 
BGM expand on a classic RBC model\footnote{See \textcite{king1999resuscitating}} by introducing monopolistic competition and endogenous firm entry for both a CES preferences and a translog preferences framework. 
Firms incur a one-time sunk entry cost to enter and then start producing after one period with constant returns (time-to-build). 
This one-time sunk entry cost is interpreted in the model as an "investment" paid in effective labor units to create a new firm/variety. Consequently, the economy has two sectors. 
A production sector where incumbent firms produce output using labor and (as an additional exercise) physical capital, and an "entry" or variety-expanding sector, where labor can be used to create new firms/varieties. 

An important aspect of the BGM framework is that each variety is interpreted as a new product line, instead of a new firm in itself. 
Therefore, whenever any reference to the number of varieties is made, it does not necessarily translate to the actual number of firms in the economy. 
The stock of varieties is a state variable that evolves with entry and exit, although in the baseline, the main driver of change is entry, 
since exit happens through a fixed probability "death shock" that affects all firms equally. Because of the sunk costs, entry is forward-looking,
that is, prospective entrants base their entry decisions on the expected present value of profits, which in this framework represents the "stock-market price" of a new firm. 

Aggregate conditions drive entry in the BGM model. In an economic expansion, higher demand and profits raise the value of starting a firm, which naturally induces more entry. 
However, due to the entry cost and, in particular, the time-to-build aspect, new firms enter sluggishly. This slow adjustment in the number of producers creates a novel 
propagation mechanism for shocks. A positive permanent productivity shock, for example, boosts output and profits immediately. However, it also sets off a gradual increase in firm entry, 
which persists even after the initial shock, given that the larger cohort of new firms continues to produce additional output, prolonging the boom. In essence, the increase in 
product variety acts like capital stock in the classic RBC model, building up slowly and sustaining production, enhancing persistence.

Another important feature is that an increase in firm entry (product variety) can affect profits at the firm level and even markups in their translog specification. 
BGM show that their framework produces procyclical profits and a procyclical number of firms, consistent with empirical evidence of net firm creation rising in booms. 
The pricing of a new firm, on the other hand, influences household savings decisions and labor allocation across sectors. If this price rises, households channel more 
labor into firm creation, increasing future variety at the cost of current output. This trade-off is similar to the one found between capital and labor in the classic 
RBC framework that BGM expand upon.

BGM find that introducing endogenous entry significantly improves the RBC model's ability to propagate shocks. The authors argue that this is due to the internal 
persistence mechanism generated by the slow build-up of new firms following a shock that the exogenous entry model lacks. Quantitatively, the model reproduces key 
second moments of macro data at least as well as the standard RBC model without the introduction of capital, and even better than it once capital is introduced. 
In particular, after the introduction of capital, the model can match the observed variance and autocorrelation of GDP seen in the data. The authors, therefore, 
conclude that abstracting away from firm entry/exit dynamics can miss important sources of persistence in business cycles. Their work laid a foundation for many 
subsequent DSGE models with entry, including applications to monetary policy\footnote{See \textcite{bilbiie2007monetary,bergin2008extensive,etro2015new}} and 
open economy/trade contexts\footnote{See \textcite{epifani2011trade,bergin2015international}}

\textcite{sedlavcek2017growth} introduce the other core framework from which I derive the firm heterogeneity and matching friction mechanisms used in this thesis. 
Sedláček and Sterk focus on how the composition of new firm cohorts varies over the business cycle and how this affects aggregate employment dynamics. 
Empirically, the authors document rich evidence using U.S. Business Dynamics Statistics for 1979-2013, finding that employment created by startup cohorts is highly
volatile and procyclical, and these cohort differences persist for many years. Additionally, they find that total employment differences across cohorts are mostly
driven by average firm size rather than by the number of startups. In other words, the authors argue that cohorts born in booms tend to contain firms that grow
larger, whereas recession-born cohorts remain smaller on average. In particular, they show that firms born during the Great Recession were not only fewer in 
number but also exhibited weaker growth potential, which in turn creates a long-lasting drag on aggregate employment at that cohort age. Sedláček and Sterk 
attribute this to selection effects (which types of businesses choose to enter in booms versus in recessions) and heterogeneous growth profiles across startups.

To explain these facts, Sedláček and Sterk develop a DSGE firm dynamics model with heterogeneous startups and entry uncertainty. Firm heterogeneity in the model arises
from differences in each startup's growth potential, modeled via the demand side. That is, new firms can choose to produce either in a low-scalability/"niche" 
environment or in a high-scalability/"mass market" environment, which is captured by the heterogeneous returns-to-scale parameters found in each environment. 
This choice at birth essentially represents a type-draw and it determines the firm's growth trajectory. After birth, firms can choose to expand their customer 
base over time by spending on marketing, subject to diminishing returns. Crucially, highly scalable startups have the incentive to invest aggressively in demand
expansion, since there are larger returns-to-scale to begin with, whereas niche firms cannot profitably scale up as much. 
The model also features an equilibrium entry condition to equalize profits (and thus entry) among different types and a friction resembling congestion in entry,
where the stock of ideas is limited and firms have to compete for entry, especially for highly scalable types.

The key mechanism driving the firm composition dynamics in the Sedláček-Sterk framework is the aggregate conditions at the time of entry. 
A favorable aggregate state disproportionally benefits mass-market firms, which rely primarily on expansion. 
Their model allows for a multitude of aggregate shock types, including preference shocks, productivity shocks, and a composition shock that shifts the distribution of business opportunities. 
They model periods of recession through a composition shock that decreases the stock of high-scalability ideas and increases the stock of 
low-scalability ideas, effectively mimicking the fact that good business opportunities are harder to come by during recessions. As a result, 
booms tend to induce a greater fraction of high-growth potential startups, whereas recessions see a higher share of niche, low-growth firms. 

This endogenous composition effect means that entry in booms is not only higher in quantity but also in quality, which translates to the 
fact that cohorts born during booms end up becoming much larger contributors to employment over time. Thus, initial conditions at birth 
have persistent effects on the path of aggregate employment/labor. In fact, their estimated model shows that most of the variation in 
cohort employment is driven by startup composition at entry, rather than post-entry shocks or decisions. Importantly, this mechanism 
operates gradually, such that the short-run impact of a boom's extra entrants is small, but those high-potential firms scale up and 
produce a long and slow-moving expansion in employment as they age. Conversely, a recession that hinders the potential of the entrant 
firms filtering out high-potential startups will lead to a persistent shortfall in employment and output, as that generation of firms will lack high-scalability startups.

Sedláček and Sterk’s model, when estimated on U.S. data, can explain several observations that are not fully addressed by other DSGE models. 
The model reproduces the fact that employment fluctuations of startups are strongly procyclical and persistent, which the authors 
attribute to the composition effect described above. Their counterfactual analysis indicates that aggregate conditions at birth 
account for the majority of the employment variation across cohorts, consistent with the idea that it is the initial selection of 
firms at birth that matters the most in long-run cohort outcomes. In the aggregate, shocks to startup composition can create large, 
low-frequency movements in employment, thus establishing a novel propagation mechanism for business cycles: cohort effects. 
Even after the shocks dissipate, the cohorts formed under its influence continue to shape growth, a phenomenon that, according to the 
authors, can help explain the slow recovery from the 2008-09 recession. Ultimately, the Sedláček-Sterk framework builds a bridge 
between business cycle research and the entrepreneurship literature, emphasizing that heterogeneity in startup quality is 
macro-relevant through theory and empirical work.

Moving on to the broader literature relevant to the topics discussed in this thesis, I start by discussing foundational work on 
incorporating endogenous firm dynamics in DSGE and RBC frameworks, and then progressively move on to more recent work. 
A first seminal contribution is given by \textcite{ghironi2007trade}, which embeds a heterogeneous-firm trade model into
a two-country dynamic GE setting. The authors show how the number of producers and product variety become state variables
affecting consumption and investment, establishing a baseline for incorporating firm entry in DSGE models. 
Subsequent research by \textcite{jaimovich2008firm} shows that net business formation is strongly procyclical 
and can itself propagate shocks. The authors present a DSGE model where an expansion induces a surge in firm entry, 
which increases competition and drives markups countercyclically lower. This mechanism causes aggregate productivity to 
rise in booms, amplifying output fluctuations. Their findings underscore an important propagation channel 
that is similarly found in the model presented in this thesis.

However, \textcite{bilbiie2012endogenous} probably represents the most influential foundational work in this vein of the literature, 
also starting a line of work that modifies their model to study optimal monetary policy with endogenous firm entry, weighing the benefits 
of variety creation against costs like entry investment and potentially inefficient boom-burst cycles of firm creation (e.g. \textcite{bilbiie2014optimal, lewis2012firm})
Another influential line of work opted for idiosyncratic firm productivity shocks alongside entry as their driver of firm heterogeneity within a DSGE structure. 
An important example is \textcite{clementi2016entry}, who build a quantitative RBC model with endogenous entry and exit, where entrants draw idiosyncratic 
productivity signals, invest, and may exit if unprofitable. A key result presented by the authors that aligns with existing literature is that introducing 
firm heterogeneity greatly increases internal propagation of aggregate shocks in the long run. Notably, earlier work by \textcite{samaniego2008entry} had 
only found modest business-cycle effects of entry in a simpler setting with homogeneous firms. In short, these foundational studies established that 
endogenous firm entry and exit dynamics, coupled with firm heterogeneity, is a potent mechanism for shock propagation. It increases volatility, 
generates endogenous persistence, and introduces new welfare considerations via product variety.

A key extension in recent years is to model heterogeneity in firms' technologies or scale, often endogenously chosen, and study how this 
affects aggregate fluctuations. Excluding \textcite{sedlavcek2017growth}, another interesting contribution to this modeling approach is given 
by \textcite{smirnyagin2023returns}. The author allows entrepreneurs to choose the returns-to-scale of their project upon entry. The resulting 
finding echoes the results in the Sedláček-Sterk framework, where high-returns-to-scale firms are less likely to be started in economic downturns.
The explanation for this phenomenon, however, now comes from firm financing, with the author arguing that large-scale projects require more 
upfront investment or financing, so when profits or credit conditions are weak, entrepreneurs defer their ambitious ideas and only smaller-scale 
firms launch. The author also states that the absence of high-growth firms can hinder recovery in recessions, further corroborating the role of 
firm heterogeneity in amplifying the propagation of shocks.

Another branch of the literature incorporates endogenous innovation, R\&D, and creative destruction into business-cycle models as both a source of firm heterogeneity and a 
new channel for technology shock propagation. Traditional RBC models take total factor productivity (TFP) as an exogenous process, but newer models allow TFP to evolve via 
the firm sector's innovation decision and diffusion of new technologies. This effectively brings Schumpeterian dynamics of innovation into the DSGE framework, enriching the 
set of propagation mechanisms.  A prominent example is the work by \textcite{comin2006medium}, and more recently \textcite{anzoategui2019endogenous}. Anzoategui et al. 
develop a medium-scale DSGE model where firms can spend on R\&D to improve the frontier technology and also incur costs to adopt existing innovations, and apply it to the 2008-09 recession. 
The authors find that the persistent productivity slowdown after the recession can, at large, be explained as an endogenous response to the drop in demand. In the model, the collapse in output 
and investment from the crisis led firms to cut R\&D and delay adoption of new technologies, causing TFP growth to falter.

On a similar note, other researchers have introduced idea production and knowledge spillovers into DSGE models. For instance,
some DSGE models feature an "innovation sector" producing new blueprints or product lines, similarly to \textcite{bilbiie2012endogenous},
but incumbent firms now decide whether to adopt new teechnologies, effectively separating variety creation from the firm dynamics.
\textcite{kung2015innovation} tie stock market valuations to R\&D-driven growth in a production economy, producing Schumpeterian dynamics: 
positive innovation shocks can cause burts of creative destruction that temporarily depresses output but raises growth later. That happens 
since a wave of new firm entry and old firm exit is unleashed by the innovation shock, leading to a reallocation of resources to a more 
optimal distribution. Conversely, and similarly to \textcite{sedlavcek2017growth}, adverse shocks can have prolonged recessionary effects
as the economy's idea pipeline dries up and less high-growth firms enter.


The frictions involved in creating new firms also play a fundamental role in dictating firm composition in an economy. There are many plausible 
sources of frictions worth considering. For instance, in the Sedláček-Sterk framework, a matching function dictates if entrants will find a matching
business idea, capturing how easily a would-be founder finds a business partner, investor or idea. \textcite{vardishvili2023entry} on the other hand,
study entrepreneurs' option to delay entry, causing a "wait-and-see" friction. The author argues that if entrepreneurs can pause plans until aggreagate
conditions improve, recessions will not only feature fewer entries, but also many delayed projects, leading to a surge of entry at the subsequent recovery.

Finally, there are welfare and policy implications to consider. In models like BGM, there is a variety externality issue, where individual firms do not internalize
that their entry adds consumer surplus by increasing variety. This might lead to under-entry in equilibrium, suggesting a possible socially inefficient equilibrium 
and arole for subsidies to startup formation. However, other potential distortions such as markups, fixed costs and financial frictioins might complicate the picture, 
not making it certain that more entry entry is always welfare-adding. Recent research by \textcite{bilbiie2019monopoly} tackles this issue by deriving optimal monetary
and fiscal policy in these settings, finding that the planner typically wants to encourage firm entry, especially during downturns, to offset the tendency of recessions
to reduce variety and future supply.

The integration of firm heterogeneity and endogenous entry into macroeconomics has opened up many questions. The fundamental debate with which I occupy myself
in this thesis is: How important is firm heterogeneity for macroeconomic fluctuations? Although \textcite{bilbiie2012endogenous} show that 
pro-cyclical startup entry amplifies business-cycle volatility, they treat all entrants as technologically identical, while \textcite{sedlavcek2017growth} 
highlight how recessions skew the composition of entrants through matching frictions but do so outside a love-of-variety RBC setting and with a different question of
firm employment in mind. By embedding Sedláček–Sterk’s technology-choice and matching block inside the BGM framework and adding a macro-level stock-of-ideas shock, 
this thesis delivers a DSGE model in which sector-specific returns-to-scale heterogeneity, entry-stage matching frictions, and endogenous idea supply shocks jointly shape 
and enrich the classic RBC shock propagation mechanisms.


%%%%%%%%%%%%%%%%%%%%%%%%%%%%%%%%%%%%%%%%%%%%%%%%%%%%%%%%%%%%%%%%%%%%%%%%%%%%%
%%%%%%%%%%%%%%%%%%%%%%%%%%%%%%%%%%%%%%%%%%%%%%%%%%%%%%%%%%%%%%%%%%%%%%%%%%%%%
%%%%%%%%%%%%%%%%%%%%%%%%%%%%%%%%%%%%%%%%%%%%%%%%%%%%%%%%%%%%%%%%%%%%%%%%%%%%%
\section{Model}
\label{sec:model}

%%%%%%%%%%%%%%%%%%%%%%%%%%%%%%%%%%%%%%%%%%%%%%%%%%%%%%%%%%%%%%%%%%%%%%%%%%%%%
%%%%%%%%%%%%%%%%%%%%%%%%%%%%%%%%%%%%%%%%%%%%%%%%%%%%%%%%%%%%%%%%%%%%%%%%%%%%%
\subsection{Baseline Bilbiie-Ghironi Framework (CES Preferences)}
\label{sec:model-bilbiie}



%%%%%%%%%%%%%%%%%%%%%%%%%%%%%%%%%%%%%%%%%%%%%%%%%%%%%%%%%%%%%%%%%%%%%%%%%%%%%
\subsection{The Sedlacek Heterogeneity Mechanism}
\label{sec:model-sedlacek}



%%%%%%%%%%%%%%%%%%%%%%%%%%%%%%%%%%%%%%%%%%%%%%%%%%%%%%%%%%%%%%%%%%%%%%%%%%%%%
%%%%%%%%%%%%%%%%%%%%%%%%%%%%%%%%%%%%%%%%%%%%%%%%%%%%%%%%%%%%%%%%%%%%%%%%%%%%%
\subsection{The Model}
\label{sec:model-mine}


%%%%%%%%%%%%%%%%%%%%%%%%%%%%%%%%%%%%%%%%%%%%%%%%%%%%%%%%%%%%%%%%%%%%%%%%%%%%%
%%%%%%%%%%%%%%%%%%%%%%%%%%%%%%%%%%%%%%%%%%%%%%%%%%%%%%%%%%%%%%%%%%%%%%%%%%%%%
%%%%%%%%%%%%%%%%%%%%%%%%%%%%%%%%%%%%%%%%%%%%%%%%%%%%%%%%%%%%%%%%%%%%%%%%%%%%%
\section{Model Solution and Dynamics}
\label{sec:solution}


%%%%%%%%%%%%%%%%%%%%%%%%%%%%%%%%%%%%%%%%%%%%%%%%%%%%%%%%%%%%%%%%%%%%%%%%%%%%%
%%%%%%%%%%%%%%%%%%%%%%%%%%%%%%%%%%%%%%%%%%%%%%%%%%%%%%%%%%%%%%%%%%%%%%%%%%%%%
\subsection{Sector–by–Sector Steady State}
\label{sec:solution-steadystate}


%%%%%%%%%%%%%%%%%%%%%%%%%%%%%%%%%%%%%%%%%%%%%%%%%%%%%%%%%%%%%%%%%%%%%%%%%%%%%
%%%%%%%%%%%%%%%%%%%%%%%%%%%%%%%%%%%%%%%%%%%%%%%%%%%%%%%%%%%%%%%%%%%%%%%%%%%%%
\subsection{Numerically Solving the Steady State}
\label{sec:solution-num}


%%%%%%%%%%%%%%%%%%%%%%%%%%%%%%%%%%%%%%%%%%%%%%%%%%%%%%%%%%%%%%%%%%%%%%%%%%%%%
%%%%%%%%%%%%%%%%%%%%%%%%%%%%%%%%%%%%%%%%%%%%%%%%%%%%%%%%%%%%%%%%%%%%%%%%%%%%%
\subsection{Properties of the Steady-State}
\label{sec:solution-properties}

%%%%%%%%%%%%%%%%%%%%%%%%%%%%%%%%%%%%%%%%%%%%%%%%%%%%%%%%%%%%%%%%%%%%%%%%%%%%%
%%%%%%%%%%%%%%%%%%%%%%%%%%%%%%%%%%%%%%%%%%%%%%%%%%%%%%%%%%%%%%%%%%%%%%%%%%%%%
%%%%%%%%%%%%%%%%%%%%%%%%%%%%%%%%%%%%%%%%%%%%%%%%%%%%%%%%%%%%%%%%%%%%%%%%%%%%%
\section{Quantitative Implementation}
\label{sec:quant}


%%%%%%%%%%%%%%%%%%%%%%%%%%%%%%%%%%%%%%%%%%%%%%%%%%%%%%%%%%%%%%%%%%%%%%%%%%%%%
%%%%%%%%%%%%%%%%%%%%%%%%%%%%%%%%%%%%%%%%%%%%%%%%%%%%%%%%%%%%%%%%%%%%%%%%%%%%%
\subsection{Calibration}
\label{sec:quant-cal} 



%%%%%%%%%%%%%%%%%%%%%%%%%%%%%%%%%%%%%%%%%%%%%%%%%%%%%%%%%%%%%%%%%%%%%%%%%%%%%
\subsection{Impulse responses}
\label{sec:quant-IRF}



%%%%%%%%%%%%%%%%%%%%%%%%%%%%%%%%%%%%%%%%%%%%%%%%%%%%%%%%%%%%%%%%%%%%%%%%%%%%%
\section{Concluding Remarks}
\label{sec:conclusion}


%%%%%%%%%%%%%%%%%%%%%%%%%%%%%%%%%%%%%%%%%%%%%%%%%%%%%%%%%%%%%%%%%%%%%%%%%%%%%
%%%%%%%%%%%%%%%%%%%%%%%%%%%%%%%%%%%%%%%%%%%%%%%%%%%%%%%%%%%%%%%%%%%%%%%%%%%%%
%%%%%%%%%%%%%%%%%%%%%%%%%%%%%%%%%%%%%%%%%%%%%%%%%%%%%%%%%%%%%%%%%%%%%%%%%%%%%
% References section
\newpage
\thispagestyle{plain}
\pagenumbering{Roman}
\printbibliography[heading=bibintoc] % Insert references

%%%%%%%%%%%%%%%%%%%%%%%%%%%%%%%%%%%%%%%%%%%%%%%%%%%%%%%%%%%%%%%%%%%%%%%%%%%%%
%%%%%%%%%%%%%%%%%%%%%%%%%%%%%%%%%%%%%%%%%%%%%%%%%%%%%%%%%%%%%%%%%%%%%%%%%%%%%
%%%%%%%%%%%%%%%%%%%%%%%%%%%%%%%%%%%%%%%%%%%%%%%%%%%%%%%%%%%%%%%%%%%%%%%%%%%%%
% Appendices section
\newpage
\begin{refsection}
\thispagestyle{plain}
\pagenumbering{arabic}  % Turn page numbering to arabic
\renewcommand*{\thepage}{A-\arabic{page}} % Add 'A' to each page number for appendices section
\addtocontents{toc}{\protect\setcounter{tocdepth}{1}} % This hides the appendix subsections in the table of contents
\begin{appendices}
%%%%%%%%%%%%%%%%%%%%%%%%%%%%%%%%%%%%%%%%%%%%%%%%%%%%%%%%%%%%%%%%%%%%%%%%%%%%%
%%%%%%%%%%%%%%%%%%%%%%%%%%%%%%%%%%%%%%%%%%%%%%%%%%%%%%%%%%%%%%%%%%%%%%%%%%%%%
\section{Additional Figures}
\label{sec-app:full}

%%%%%%%%%%%%%%%%%%%%%%%%%%%%%%%%%%%%%%%%%%%%%%%%%%%%%%%%%%%%%%%%%%%%%%%%%%%%%
\section{Computational Implementation}
\label{sec-app:codes}



%%%%%%%%%%%%%%%%%%%%%%%%%%%%%%%%%%%%%%%%%%%%%%%%%%%%%%%%%%%%%%%%%%%%%%%%%%%%%
\section{Data for Model Calibration}
\label{sec-app:data}


%%%%%%%%%%%%%%%%%%%%%%%%%%%%%%%%%%%%%%%%%%%%%%%%%%%%%%%%%%%%%%%%%%%%%%%%%%%%%
\section{Supplementary Results}
\label{sec-app:figures}

This appendix contains supplementary figures and tables, placed here for the sake of brevity in the main text. It is ordered according to the corresponding sections in the text.



\newpage
%%%%%%%%%%%%%%%%%%%%%%%%%%%%%%%%%%%%%%%%%%%%%%%%%%%%%%%%%%%%%%%%%%%%%%%%%%%%%
%%%%%%%%%%%%%%%%%%%%%%%%%%%%%%%%%%%%%%%%%%%%%%%%%%%%%%%%%%%%%%%%%%%%%%%%%%%%%
\end{appendices}
\thispagestyle{plain}
\pagenumbering{Roman} % Start Roman page numbering for appendix references
\renewcommand*{\thepage}{A-\Roman{page}} % Add 'A' to each page number for appendix references

\addcontentsline{toc}{section}{Appendix References}
\printbibliography[heading=subbibliography, title={Appendix References}]
\thispagestyle{plain}
\cleardoublepage % Ensure the next content starts on an odd page (if your document is two-sided)

\end{refsection}

%%%%%%%%%%%%%%%%%%%%%%%%%%%%%%%%%%%%%%%%%%%%%%%%%%%%%%%%%%%%%%%%%%%%%%%%%%%%%
%%%%%%%%%%%%%%%%%%%%%%%%%%%%%%%%%%%%%%%%%%%%%%%%%%%%%%%%%%%%%%%%%%%%%%%%%%%%%
%%%%%%%%%%%%%%%%%%%%%%%%%%%%%%%%%%%%%%%%%%%%%%%%%%%%%%%%%%%%%%%%%%%%%%%%%%%%%
\newpage
\thispagestyle{plain}
\pagenumbering{gobble} % Turn page numbering off
\section*{Statement of Authorship} % Include statement of authorship
I hereby confirm that the work presented has been performed and interpreted solely by myself except for where I explicitly identified the contrary. I assure that this work has not been presented in any other form for the fulfillment of any other degree or qualification. Ideas taken from other works in letter and in spirit are identified in every single case.

\vspace{2cm}
\noindent
\rule{8cm}{0.4pt}\\
Enrico Truzzi\\
Bonn, the \printdate{2025-7-1}
\end{document}
% End Document
%%%%%%%%%%%%%%%%%%%%%%%%%%%%%%%%%%%%%%%%%%%%%%%%%%%%%%%%%%%%%%%%%%%%%%%%%%%%%
%%%%%%%%%%%%%%%%%%%%%%%%%%%%%%%%%%%%%%%%%%%%%%%%%%%%%%%%%%%%%%%%%%%%%%%%%%%%%
%%%%%%%%%%%%%%%%%%%%%%%%%%%%%%%%%%%%%%%%%%%%%%%%%%%%%%%%%%%%%%%%%%%%%%%%%%%%%
%%%%%%%%%%%%%%%%%%%%%%%%%%%%%%%%%%%%%%%%%%%%%%%%%%%%%%%%%%%%%%%%%%%%%%%%%%%%%
%%%%%%%%%%%%%%%%%%%%%%%%%%%%%%%%%%%%%%%%%%%%%%%%%%%%%%%%%%%%%%%%%%%%%%%%%%%%%
%%%%%%%%%%%%%%%%%%%%%%%%%%%%%%%%%%%%%%%%%%%%%%%%%%%%%%%%%%%%%%%%%%%%%%%%%%%%%
