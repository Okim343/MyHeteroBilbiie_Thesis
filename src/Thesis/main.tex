\documentclass[a4paper,12pt]{article} % Set documentclass article and set font size

%%%%%%%%%%%%%%%%%%%%%%%%%%%%%%%%%%%%%%%%%%%%%%%%%%%%%%%%%%%%%%%%%%%%%%%%%%%%%
%%%%%%%%%%%%%%%%%%%%%%%%%%%%%%%%%%%%%%%%%%%%%%%%%%%%%%%%%%%%%%%%%%%%%%%%%%%%%
%%%%%%%%%%%%%%%%%%%%%%%%%%%%%%%%%%%%%%%%%%%%%%%%%%%%%%%%%%%%%%%%%%%%%%%%%%%%%
% Packages

\usepackage[utf8]{inputenc}
\usepackage[british]{babel}
\usepackage[british]{isodate}
\usepackage{csquotes}
\usepackage{mathtools}
\usepackage{amssymb}
\usepackage{geometry}
\usepackage{tabularx}
\usepackage[
backend=biber,
style=apa,
sorting=nyt, % Sort by name, year, title
backref=true,
dateabbrev=false,
language=british,
urldate=long
]{biblatex}
\usepackage{xcolor}
\usepackage[toc,page]{appendix}
\usepackage{svg}
\usepackage{float}
\usepackage{longtable}
\usepackage{multirow}
\usepackage{multirow,multicol,makecell,booktabs}
\usepackage[justification=centering]{caption}
\usepackage{subcaption}
\usepackage{setspace}
\usepackage{ragged2e}
\usepackage{fancyhdr}
\usepackage{enumitem}
\usepackage{tikz}
\usepackage{rotating}
\usepackage{hyperref} % Load after everything else
\usepackage{graphicx}
\usepackage[symbols, nopostdot, automake, nonumberlist]{glossaries}
\usepackage{booktabs,subcaption}


%%%%%%%%%%%%%%%%%%%%%%%%%%%%%%%%%%%%%%%%%%%%%%%%%%%%%%%%%%%%%%%%%%%%%%%%%%%%%
%%%%%%%%%%%%%%%%%%%%%%%%%%%%%%%%%%%%%%%%%%%%%%%%%%%%%%%%%%%%%%%%%%%%%%%%%%%%%
%%%%%%%%%%%%%%%%%%%%%%%%%%%%%%%%%%%%%%%%%%%%%%%%%%%%%%%%%%%%%%%%%%%%%%%%%%%%%
% Custom Settings

\addbibresource{master_thesis.bib} % Add bibliography source

\setlength\bibitemsep{1.5\itemsep} % Adjust distance of reference entries

\renewcommand*\appendixpagename{\Large Appendices} % Adjust size of title of the appendices section

\renewcommand*{\finentrypunct}{}
\renewbibmacro*{pageref}{%
  \addperiod% NEW
  \iflistundef{pageref}
    {}
    {\newline\footnotesize\printtext[parens]{%
       \ifnumgreater{\value{pageref}}{1}
         {\bibstring{backrefpages}\ppspace}
     {\bibstring{backrefpage}\ppspace}%
       \printlist[pageref][-\value{listtotal}]{pageref}}}}%
\DefineBibliographyStrings{english}{
  backrefpage={Cited on page},
  backrefpages={Cited on pages}
}

\geometry{ % Adjust paper geometry
 a4paper,
 right=25mm,
 bottom=20mm,
 left=25mm,
 top=25mm}
\urlstyle{same} % Make URL font the same with the rest of the document
\onehalfspacing % Adjust spacing
%\setstretch{1.75}
%\doublespacing
\allowdisplaybreaks % Allow for breaks in equations across pages
\addto\captionsbritish{ % Replace "english" with the language you use
  \renewcommand{\contentsname}%
    {Table of Contents}%
} % Adjust title of the table of contents
\numberwithin{equation}{section} % Number equations with sections
\numberwithin{figure}{section}
\numberwithin{table}{section}
%\setcounter{tocdepth}{2} % Set maximum level to which sections are displayed in the table of contents

\newcommand{\appendixsubsection}[1]{
    \stepcounter{subsection}
    \subsection*{\Alph{section}.\arabic{subsection}\hspace{1em}{#1}}
}

\newcommand{\dat}{\textbf}      % Data column style
\newcommand{\bil}{\relax}       % Bilbiie column (roman)
\newcommand{\rbc}{\emph}        % RBC column style
\newcommand{\mine}[1]{\underline{#1}}   % underline my model values


\pagestyle{plain}

% Define a custom name format that prints names of authors in references in bold
\DeclareNameWrapperFormat{sortname}{\mkbibbold{#1}}
\DeclareNameWrapperAlias{author}{sortname}

% Add definition format
\newtheorem{definition}{Definition}

% Make links of glossary entries black
\makeatletter
\newcommand*{\glsplainhyperlink}[2]{%
    \begingroup%
      \hypersetup{hidelinks}%
      \hyperlink{#1}{#2}%
    \endgroup%
}
\let\@glslink\glsplainhyperlink
\makeatother

%%%%%%%%%%%%%%%%%%%%%%%%%%%%%%%%%%%%%%%%%%%%%%%%%%%%%%%%%%%%%%%%%%%%%%%%%%%%%
%%%%%%%%%%%%%%%%%%%%%%%%%%%%%%%%%%%%%%%%%%%%%%%%%%%%%%%%%%%%%%%%%%%%%%%%%%%%%
%%%%%%%%%%%%%%%%%%%%%%%%%%%%%%%%%%%%%%%%%%%%%%%%%%%%%%%%%%%%%%%%%%%%%%%%%%%%%
% Glossary
\makeglossaries 

% Glossary entries
\newglossaryentry{nk}{
name=NK,
description={New Keynesian}
}

\newglossaryentry{ha}{
name=HA,
description={Heterogeneous Agent(s)}
}

\newglossaryentry{hank}{
name=HANK,
description={Heterogeneous Agent New Keynesian}
}

\newglossaryentry{gfc}{
name=GFC,
description={Great Financial Crisis}
}

\newglossaryentry{ltv}{
name=LTV,
description={Loan-to-Value}
}

\newglossaryentry{dsge}{
name=DSGE,
description={Dynamic Stochastic General Equilibrium}
}

\newglossaryentry{egm}{
name=EGM,
description={Endogenous Gridpoints Method}
}

\newglossaryentry{iou}{
name=IOU,
description={I Owe You}
}

\newglossaryentry{ghh}{
name=GHH,
description={Greenwood-Hercowitz-Huffman (\cite{ghh1988})}
}

\newglossaryentry{ar}{
name=AR,
description={Autoregressive}
}

\newglossaryentry{zlb}{
name=ZLB,
description={Zero Lower Bound}
}

\newglossaryentry{mpc}{
name=MPC,
description={Marginal Propensity to Consume}
}

\newglossaryentry{fof}{
name=FoF,
description={Flow of Funds}
}

\newglossaryentry{nkpc}{
name=NKPC,
description={New Keynesian Phillips Curve}
}

\newglossaryentry{foc}{
name=FOC,
description={First Order Condition}
}

\newglossaryentry{irf}{
name=IRF,
description={Impulse Response Function}
}

\newglossaryentry{crra}{
name=CRRA,
description={Constant Relative Risk Aversion}
}

\newglossaryentry{gdp}{
name=GDP,
description={Gross Domestic Product}
}

%%%%%%%%%%%%%%%%%%%%%%%%%%%%%%%%%%%%%%%%%%%%%%%%%%%%%%%%%%%%%%%%%%%%%%%%%%%%%
%%%%%%%%%%%%%%%%%%%%%%%%%%%%%%%%%%%%%%%%%%%%%%%%%%%%%%%%%%%%%%%%%%%%%%%%%%%%%
%%%%%%%%%%%%%%%%%%%%%%%%%%%%%%%%%%%%%%%%%%%%%%%%%%%%%%%%%%%%%%%%%%%%%%%%%%%%%
% hyperref package
\usepackage{footnotebackref} % Brings you back to where the footnote is in the text
\hypersetup{ 
    colorlinks,
    linkcolor=blue,
    filecolor=blue,  
    citecolor=blue,
    urlcolor=blue} % Setup with the hyperref package

%%%%%%%%%%%%%%%%%%%%%%%%%%%%%%%%%%%%%%%%%%%%%%%%%%%%%%%%%%%%%%%%%%%%%%%%%%%%%
%%%%%%%%%%%%%%%%%%%%%%%%%%%%%%%%%%%%%%%%%%%%%%%%%%%%%%%%%%%%%%%%%%%%%%%%%%%%%
%%%%%%%%%%%%%%%%%%%%%%%%%%%%%%%%%%%%%%%%%%%%%%%%%%%%%%%%%%%%%%%%%%%%%%%%%%%%%
% Formalities
\title{Technology Choice and Firm Heterogeneity in an RBC Model with Endogenous Entry}
\author{}
\date{}

%%%%%%%%%%%%%%%%%%%%%%%%%%%%%%%%%%%%%%%%%%%%%%%%%%%%%%%%%%%%%%%%%%%%%%%%%%%%%
%%%%%%%%%%%%%%%%%%%%%%%%%%%%%%%%%%%%%%%%%%%%%%%%%%%%%%%%%%%%%%%%%%%%%%%%%%%%%
%%%%%%%%%%%%%%%%%%%%%%%%%%%%%%%%%%%%%%%%%%%%%%%%%%%%%%%%%%%%%%%%%%%%%%%%%%%%%
%%%%%%%%%%%%%%%%%%%%%%%%%%%%%%%%%%%%%%%%%%%%%%%%%%%%%%%%%%%%%%%%%%%%%%%%%%%%%
%%%%%%%%%%%%%%%%%%%%%%%%%%%%%%%%%%%%%%%%%%%%%%%%%%%%%%%%%%%%%%%%%%%%%%%%%%%%%
%%%%%%%%%%%%%%%%%%%%%%%%%%%%%%%%%%%%%%%%%%%%%%%%%%%%%%%%%%%%%%%%%%%%%%%%%%%%%
% Begin Document
\begin{document}
\selectlanguage{british}
\pagenumbering{gobble} % Turn page numbering off
\maketitle % Create title

\vspace{4cm}
\begin{center}
Master Thesis Presented to the\\
Department of Economics at the\\
Rheinische Friedrich-Wilhelms-Universität Bonn\\
\vspace{1cm}
In Partial Fulfillment of the Requirements for the Degree of\\
Master of Science (M.Sc.)
\end{center}

\vspace{5cm}
\begin{center}
\begin{tabular}{ l l }
Supervisors: & Prof.~Dr.~Christian Bayer \\
& Dr.~Zheng Gong \\
& \\
& \\
Date of Submission: & \printdate{2025-7-1}\\
Author: & Enrico Truzzi\\
Matriculation Number: & 50182495
\end{tabular}
\end{center}

%%%%%%%%%%%%%%%%%%%%%%%%%%%%%%%%%%%%%%%%%%%%%%%%%%%%%%%%%%%%%%%%%%%%%%%%%%%%%
%%%%%%%%%%%%%%%%%%%%%%%%%%%%%%%%%%%%%%%%%%%%%%%%%%%%%%%%%%%%%%%%%%%%%%%%%%%%%
%%%%%%%%%%%%%%%%%%%%%%%%%%%%%%%%%%%%%%%%%%%%%%%%%%%%%%%%%%%%%%%%%%%%%%%%%%%%%
\newpage
\pagenumbering{roman} % Turn page numbering to small roman
\setcounter{tocdepth}{2}
{ \hypersetup{hidelinks} \tableofcontents } % Hide links in table of contents

\newpage
{ \hypersetup{hidelinks} \listoffigures } % Hide links in list of figures

\newpage
{ \hypersetup{hidelinks} \listoftables } % Hide links in list of tables

% Add here list of acronyms
\newpage
\setlist[description]{leftmargin=!, labelwidth=5em} % Change for glossaries
\printglossary[title=List of Acronyms]
\setlist[description]{style=standard} % Reset settings back to default


%%%%%%%%%%%%%%%%%%%%%%%%%%%%%%%%%%%%%%%%%%%%%%%%%%%%%%%%%%%%%%%%%%%%%%%%%%%%%
%%%%%%%%%%%%%%%%%%%%%%%%%%%%%%%%%%%%%%%%%%%%%%%%%%%%%%%%%%%%%%%%%%%%%%%%%%%%%
%%%%%%%%%%%%%%%%%%%%%%%%%%%%%%%%%%%%%%%%%%%%%%%%%%%%%%%%%%%%%%%%%%%%%%%%%%%%%
\newpage
\pagenumbering{arabic} % Turn page numbering to small arabic
\pagestyle{fancy}
\fancyhf{}
\fancyhead[C]{\leftmark}
\fancyfoot[C]{\thepage}

%%%%%%%%%%%%%%%%%%%%%%%%%%%%%%%%%%%%%%%%%%%%%%%%%%%%%%%%%%%%%%%%%%%%%%%%%%%%%
%%%%%%%%%%%%%%%%%%%%%%%%%%%%%%%%%%%%%%%%%%%%%%%%%%%%%%%%%%%%%%%%%%%%%%%%%%%%%
%%%%%%%%%%%%%%%%%%%%%%%%%%%%%%%%%%%%%%%%%%%%%%%%%%%%%%%%%%%%%%%%%%%%%%%%%%%%%
\thispagestyle{plain}
\section{Introduction}
\label{sec:introduction}

The heterogeneity of producers, and not only their number, represents an important feature of any economy. Firms can diverge on many aspects,
but a fundamental one is scalability. On one hand, more scalable firms tend to grow to larger sizes due to high returns to scale while less scalable firms tend
to expand less. According to a technical report by the European Commission, high-growth enterprises account for just 11\% of EU firms but created 53\% 
of net jobs in 2015-16 (\cite{flachenecker2020highgrowth}).
On the other hand, highly scalable and lucrative business ideas tend also to invoke higher competition, leading to
substantial friction when establishing a firm for prospective entrepreneurs. \textcite{sedlavcek2017growth} use data from the 
Business Dynamics Statistics (BDS) US-census to show that, depending on how scalable the idea is, the probability of 
starting a successful business can range from 0.2\% all the way to 62\%.

Heterogenous scaling within firm cohorts also leads to heterogeneous responses to shocks, which can yield novel aggregate shock dynamics when compared
with standard responses. Using census microdata, \textcite{smirnyagin2023returns} shows that the entry rate of high returns-to-scale firms falls 
twice as much as for low returns-to-scale firms during recessions. \textcite{calligaris2023employment} demonstrate that employment losses in 2020 from the COVID-19
pandemic were significantly larger in low-telework sectors incapable of quick scaling and adaptation.

Hence, this thesis investigates how shock responses in a Real-Business-Cycle (RBC) model with endogenous entry are affected by the introduction of across-sector 
differences in returns-to-scale parameters and entry-matching frictions. In particular, how do sector-specific returns-to-scale and matching frictions at birth 
alter the propagation of aggregate shocks, especially when compared to a baseline RBC framework? I also investigate the impact of a sudden increase in business 
opportunities across the economy, and how firm-level and aggregate-level responses to this shock diverge.

By doing so, I integrate the seminal RBC with endogenous entry framework found in \textcite{bilbiie2012endogenous} (henceforth BGM) with the technology-choice mechanism from
\textcite{sedlavcek2017growth} in one DSGE environment with multiple production sectors\footnote{In my framework, each sector corresponds to one 
returns-to-scale parameter, such that a symmetric within-sector equilibrium makes all firms that belong to the same sector share equal prices and profits.}. 
Additionally, a novel stock-of-ideas shock is introduced by modifying the composition shock found in the 
Sedláček-Sterk framework, to capture the effect of a boom in business viability in an endogenous and micro-consistent way. This approach allows the study of
firm entry and heterogeneity dynamics inside a pre-existing and well-studied RBC framework, delivering new insights for industrial policy backing 
production activities in recession periods that takes scaling differences into consideration.

For instance, I find that by introducing firm heterogeneity in returns-to-scale parameters, the impact of a productivity shock on aggregate output 
substantially increases when compared to the baseline RBC framework in BGM. This new magnitude comes from two new shock transmission channels: intersectoral 
labor reallocation and demand reallocation across sectors; which are not captured by the standard representative firm baseline. Additionally,
I find a negative firm-level impact across different technology types of positive productivity and stock-of-ideas shocks, mainly caused by over-entry
and a subsequent competition glut. This effect drives firm-level profits down despite sectoral aggregates going up from higher productivity and/or 
a higher number of incumbents.

An important point is that, as in the original BGM framework, each producer is equated to an individual product line, accounting for the fact that new varieties
are also added by existing firms. With this in mind, analyzing steady-state firm behavior pins down interesting facts about the model's functioning. 
The relationship between firm-level output and labor demand, for instance, is highly connected to the returns-to-scale parameter. The larger the 
parameter, the longer a firm can expand before hitting diminishing returns, so highly scalable sectors naturally hold more production lines. On the 
competition front, any parameter change that depresses entry enlarges the slice of the market served by each surviving producer. A congestion channel also
exists, such that prospective incumbents face decreasing "returns-to-entry", where success odds tumble in a non-linear way going from low- to high-scalability sectors.
Finally, in steady-state all entry happens solely to rebuild the stock of varieties, such that entry and dividends work together much like depreciation and
investment in a capital-based RBC model.


Recent literature has been increasingly concerned with the impact of firm heterogeneity on the transmission and persistence of shocks. More foundational works 
like BGM and \textcite{ghironi2007trade} have given substantial attention to endogeneizing entry dynamics as a way to explain the pro-cyclicality of net-entry in 
business cycles.
Another strand of work focuses on technological heterogeneity and cohort effects as potential explanations for the persistent effects of recessions on startup cohorts
(See \cite{sedlavcek2017growth,clementi2016entry}). Finally, more recent models base their dynamics solely on innovation/idea-driven mechanisms, exploring
new transmission channels that arise from R\&D activity (See \cite{anzoategui2019endogenous,comin2006medium}). This thesis bridges these strands by allowing firm technology choice to interact with variety creation
in a standard RBC skeleton with endogenous entry.

The insights presented here are of value to the design of policies such as startup subsidies, macro-prudential entry policies, industrial strategy, and others.
For instance, facilitating conditions for early-stage entrepreneurs can have significant impacts on active firms, especially in highly attractive sectors that are
hit the hardest by competition overflow. Policymakers must adjust their subsidies to make sure that the same few large firms that employ most of the economy do not
suffer too hard of a blow from a positive shock that incentivizes entry. Another important discussion comes from potential externalities in variety creation in the baseline BGM 
model, where a new entrant raises welfare through additional varieties, but its private incentive equals the markup, not the social value of the variety 
(See \cite{bilbiie2019monopoly}). In my framework, it could be the case that across-sector returns-to-scale dispersion could generate allocative inefficiency, 
as is the case with markup dispersion (See \cite{baqaee2020productivity}), or amplify congestion 
externality\footnote{ See \textcite{bilbiie2014optimal}, where the authors show that Ramsey-optimal policy in the BGM framework departs from
price stability precisely to offset the congestion/variety wedges, where more entrants raise marginal entry costs for all and lower existing profits, a 
negative pecuniary externality absent from firms' decisions.} in highly scalable sectors. 

The remainder of this thesis is organized as follows. Section \ref{sec:literature} begins with an in-depth review of two central frameworks 
to this work: \textcite{bilbiie2012endogenous} and \textcite{sedlavcek2017growth}. This is followed by a brief overview of the broader literature 
on firm heterogeneity and the transmission of shocks. Section \ref{sec:model} introduces the key elements of the BGM framework and the technology 
choice and entry mechanism of Sedláček and Sterk, culminating in the full model that integrates both. Section \ref{sec:solution} outlines the 
solution method, detailing the algorithm used to compute the model’s steady state and examining its steady-state properties, with a focus on firm 
behavior. Section \ref{sec:quant} presents the model calibration and analyzes its impulse responses to both a standard productivity shock and a 
stock-of-ideas shock. Finally, Section \ref{sec:conclusion} concludes with the main findings.


%%%%%%%%%%%%%%%%%%%%%%%%%%%%%%%%%%%%%%%%%%%%%%%%%%%%%%%%%%%%%%%%%%%%%%%%%%%%%
%%%%%%%%%%%%%%%%%%%%%%%%%%%%%%%%%%%%%%%%%%%%%%%%%%%%%%%%%%%%%%%%%%%%%%%%%%%%%
%%%%%%%%%%%%%%%%%%%%%%%%%%%%%%%%%%%%%%%%%%%%%%%%%%%%%%%%%%%%%%%%%%%%%%%%%%%%%
\section{Related Literature}
\label{sec:literature}



I first start this review of the literature by addressing \textcite{bilbiie2012endogenous}, since it represents the core framework behind the model developed in this thesis. 
BGM expand on a classic RBC model (See \cite{campbell1994inspecting}) by introducing monopolistic competition and endogenous firm entry for both a C.E.S. and a translog preferences framework. 
Firms incur a one-time sunk entry cost to enter and then start producing after one period with constant returns (time-to-build). 
This one-time sunk entry cost is interpreted in the model as an "investment" paid in effective labor units to create a new firm/variety. Consequently, the economy has two sectors. 
A production sector where incumbent firms produce output using labor and (as an additional exercise) physical capital, and an "entry" or variety-expanding sector, where labor can be used to create new firms/varieties. 

An important aspect of the BGM framework is that each variety is interpreted as a new product line, instead of a new firm in itself. 
Therefore, whenever any reference to the number of varieties is made, it does not necessarily translate to the actual number of firms in the economy. 
The stock of varieties is a state variable that evolves with entry and exit, although in the baseline, the main driver of change is entry 
since exit happens through a fixed probability "death shock" that affects all firms equally. Because of the sunk costs, entry is forward-looking,
that is, prospective entrants base their entry decisions on the expected present value of profits, which in this framework represents the "stock-market price" of a new firm. 

Aggregate conditions drive entry in the BGM model. In an economic expansion, higher demand and profits raise the value of starting a firm, which naturally induces more entry. 
However, due to the entry cost and, in particular, the time-to-build aspect, new firms enter sluggishly. This slow adjustment in the number of producers creates a novel 
propagation mechanism for shocks. A positive permanent productivity shock, for example, boosts output and profits immediately. However, it also sets off a gradual increase in firm entry, 
which persists even after the initial shock, given that the larger cohort of new firms continues to produce additional output, prolonging the boom. In essence, the increase in 
product variety acts like capital stock in the classic RBC model, building up slowly and sustaining production, enhancing persistence.

Another important feature is that an increase in firm entry (product variety) can affect profits at the firm level and even markups in their translog specification. 
BGM show that their framework produces procyclical profits and a procyclical number of firms, consistent with empirical evidence of net firm creation rising in booms. 
The pricing of a new firm, on the other hand, influences household savings decisions and labor allocation across sectors. If this price rises, households channel more 
labor into firm creation, increasing future variety at the cost of current output. This trade-off is similar to the one found between capital and labor in the classic 
RBC framework that BGM expand upon.

BGM find that introducing endogenous entry significantly improves the RBC model's ability to propagate shocks. The authors argue that this is due to the internal 
persistence mechanism generated by the slow build-up of new firms following a shock that the exogenous entry model lacks. Quantitatively, the model reproduces key 
second moments of macro data at least as well as the standard RBC model without the introduction of capital, and even better than that once capital is introduced. 
In particular, after the introduction of capital, the model can match the observed variance and autocorrelation of GDP seen in the data. The authors, therefore, 
conclude that abstracting away from firm entry/exit dynamics can miss important sources of persistence in business cycles. Their work laid a foundation for many 
subsequent DSGE models with entry, including applications to monetary policy\footnote{See \textcite{bilbiie2007monetary,bergin2008extensive,etro2015new}} and 
open economy/trade contexts\footnote{See \textcite{epifani2011trade,bergin2015international}}

\textcite{sedlavcek2017growth} introduce the other core framework from which I derive the firm heterogeneity and matching friction mechanisms used in this thesis. 
Sedláček and Sterk focus on how the composition of new firm cohorts varies over the business cycle and how this affects aggregate employment dynamics. 
Empirically, the authors document rich evidence using U.S. Business Dynamics Statistics for 1979-2013, finding that employment created by startup cohorts is highly
volatile and procyclical, and these cohort differences persist for many years. Additionally, they find that total employment differences across cohorts are mostly
driven by average firm size rather than by the number of startups. In other words, the authors argue that cohorts born in booms tend to contain firms that grow
larger, whereas recession-born cohorts remain smaller on average. In particular, they show that firms born during the Great Recession were not only fewer in 
number but also exhibited weaker growth potential, which in turn creates a long-lasting drag on aggregate employment at that cohort age. Sedláček and Sterk 
attribute this to selection effects (which types of businesses choose to enter in booms versus in recessions) and heterogeneous growth profiles across startups.

To explain these facts, Sedláček and Sterk develop a DSGE firm dynamics model with heterogeneous startups and entry uncertainty. Firm heterogeneity in the model arises
from differences in each startup's growth potential, modeled via the demand side. That is, new firms can choose to produce either in a low-scalability/"niche" 
environment or in a high-scalability/"mass market" environment, which is captured by the heterogeneous returns-to-scale parameters found in each environment. 
This choice at birth essentially represents a type-draw, and it determines the firm's growth trajectory. After birth, firms can choose to expand their customer 
base over time by spending on marketing, subject to diminishing returns. Crucially, highly scalable startups have the incentive to invest aggressively in demand
expansion, since there are larger returns-to-scale to begin with, whereas niche firms cannot profitably scale up as much. 
The model also features an equilibrium entry condition to equalize profits (and thus entry) among different types and a friction resembling congestion in entry,
where the stock of ideas is limited and firms have to compete for entry, especially for highly scalable types.

The key mechanism driving the firm composition dynamics in the Sedláček-Sterk framework is the aggregate conditions at the time of entry. 
A favorable aggregate state disproportionally benefits mass-market firms, which rely primarily on expansion. 
Their model allows for a multitude of aggregate shock types, including preference shocks, productivity shocks, and a composition shock that shifts the distribution of business opportunities. 
They model periods of recession through a composition shock that decreases the stock of high-scalability ideas and increases the stock of 
low-scalability ideas, effectively mimicking the fact that good business opportunities are harder to come by during recessions. As a result, 
booms tend to induce a greater fraction of high-growth potential startups, whereas recessions see a higher share of niche, low-growth firms. 

This endogenous composition effect means that entry in booms is not only higher in quantity but also in quality, which translates to the 
fact that cohorts born during booms end up becoming much larger contributors to employment over time. Thus, initial conditions at birth 
have persistent effects on the path of aggregate employment/labor. In fact, their estimated model shows that most of the variation in 
cohort employment is driven by startup composition at entry, rather than post-entry shocks or decisions. Importantly, this mechanism 
operates gradually, such that the short-run impact of a boom's extra entrants is small, but those high-potential firms scale up and 
produce a long and slow-moving expansion in employment as they age. Conversely, a recession that hinders the potential of the entrant 
firms filtering out high-potential startups will lead to a persistent shortfall in employment and output, as that generation of firms will lack 
high-scalability startups.

Sedláček and Sterk’s model, when estimated on U.S. data, can explain several observations that are not fully addressed by other DSGE models. 
The model reproduces the fact that employment fluctuations of startups are strongly procyclical and persistent, which the authors 
attribute to the composition effect described above. Their counterfactual analysis indicates that aggregate conditions at birth 
account for the majority of the employment variation across cohorts, consistent with the idea that it is the initial selection of 
firms at birth that matters the most in long-run cohort outcomes. In the aggregate, shocks to startup composition can create large, 
low-frequency movements in employment, thus establishing a novel propagation mechanism for business cycles: cohort effects. 
Even after the shocks dissipate, the cohorts formed under its influence continue to shape growth, a phenomenon that, according to the 
authors, can help explain the slow recovery from the 2008-09 recession. Ultimately, the Sedláček-Sterk framework builds a bridge 
between business cycle research and entrepreneurship literature, emphasizing that heterogeneity in startup quality is 
macro-relevant through theory and empirical work.

Moving on to the broader literature relevant to the topics discussed in this thesis, I start by discussing foundational work on 
incorporating endogenous firm dynamics in DSGE and RBC frameworks, and then progressively move on to more recent work. 
A first seminal contribution is given by \textcite{ghironi2007trade}, which embeds a heterogeneous-firm trade model into
a two-country dynamic GE setting. The authors show how the number of producers and product variety become state variables
affecting consumption and investment, establishing a baseline for incorporating firm entry in DSGE models. 
Subsequent research by \textcite{jaimovich2008firm} shows that net business formation is strongly procyclical 
and can itself propagate shocks. The authors present a DSGE model where an expansion induces a surge in firm entry, 
which increases competition and drives markups countercyclically lower. This mechanism causes aggregate productivity to 
rise in booms, amplifying output fluctuations. Their findings underscore an important propagation channel 
that is similarly found in the model presented in this thesis.

However, \textcite{bilbiie2012endogenous} probably represent the most influential foundational work in this vein of the literature, 
also starting a line of work that modifies their model to study optimal monetary policy with endogenous firm entry, weighing the benefits 
of variety creation against costs like entry investment and potentially inefficient boom-burst cycles of firm creation (e.g. \cite{bilbiie2014optimal, lewis2012firm})
Another influential line of work opted for idiosyncratic firm productivity shocks alongside entry as their driver of firm heterogeneity within a DSGE structure. 
An important example is \textcite{clementi2016entry}, who build a quantitative RBC model with endogenous entry and exit, where entrants draw idiosyncratic 
productivity signals, invest, and may exit if unprofitable. A key result presented by the authors that aligns with existing literature is that introducing 
firm heterogeneity greatly increases the internal propagation of aggregate shocks in the long run. Notably, earlier work by \textcite{samaniego2008entry} had 
only found modest business-cycle effects of entry in a simpler setting with homogeneous firms. In short, these foundational studies established that 
endogenous firm entry and exit dynamics, coupled with firm heterogeneity, is a potent mechanism for shock propagation. It increases volatility, 
generates endogenous persistence, and introduces new welfare considerations via product variety.

A key extension in recent years is to model heterogeneity in firms' technologies or scale, often endogenously chosen, and study how this 
affects aggregate fluctuations. Excluding \textcite{sedlavcek2017growth}, another interesting contribution to this modeling approach is given 
by \textcite{smirnyagin2023returns}. The author allows entrepreneurs to choose the returns-to-scale of their project upon entry. The resulting 
finding echoes the results in the Sedláček-Sterk framework, where high-returns-to-scale firms are less likely to be started in economic downturns.
The explanation for this phenomenon, however, now comes from firm financing, with the author arguing that large-scale projects require more 
upfront investment or financing, so when profits or credit conditions are weak, entrepreneurs defer their ambitious ideas and only smaller-scale 
firms launch. The author also states that the absence of high-growth firms can hinder recovery in recessions, further corroborating the role of 
firm heterogeneity in amplifying the propagation of shocks.

Another branch of the literature incorporates endogenous innovation, R\&D, and creative destruction into business-cycle models as both a source of firm heterogeneity and a 
new channel for technology shock propagation. Traditional RBC models take total factor productivity (TFP) as an exogenous process, but more recent models allow TFP to evolve via 
the firm sector's innovation-decision and diffusion of new technologies. This effectively brings Schumpeterian dynamics of innovation into the DSGE framework, enriching the 
set of propagation mechanisms.  A prominent example is the work by \textcite{comin2006medium}, and more recently \textcite{anzoategui2019endogenous}. Anzoategui et al. 
develop a medium-scale DSGE model where firms can spend on R\&D to improve the frontier technology and also incur costs to adopt existing innovations, and apply it to the 2008-09 recession. 
The authors find that the persistent productivity slowdown after the recession can, at large, be explained as an endogenous response to the drop in demand. In the model, the collapse in output 
and investment from the crisis led firms to cut R\&D and delay the adoption of new technologies, causing TFP growth to falter.

On a similar note, other researchers have introduced idea production and knowledge spillovers into DSGE models. For instance,
some DSGE models feature an "innovation sector" producing new blueprints or product lines, similar to \textcite{bilbiie2012endogenous},
but incumbent firms now decide whether to adopt new technologies, effectively separating variety creation from firm dynamics.
\textcite{kung2015innovation} tie stock market valuations to R\&D-driven growth in a production economy, producing Schumpeterian dynamics: 
positive innovation shocks can cause bursts of creative destruction that temporarily depress output but raise growth later. That happens 
since a wave of new firm entry and old firm exit is unleashed by the innovation shock, leading to a reallocation of resources to a more 
optimal distribution. Conversely, and similarly to \textcite{sedlavcek2017growth}, adverse shocks can have prolonged recessionary effects
as the economy's idea pipeline dries up and fewer high-growth firms enter.


The frictions involved in creating new firms also play a fundamental role in dictating firm composition in an economy. There are many plausible 
sources of frictions worth considering. For instance, in the Sedláček-Sterk framework, a matching function dictates if entrants will find a matching
business idea, capturing how easily a would-be founder finds a business partner, investor, or idea. \textcite{vardishvili2023entry} on the other hand,
study entrepreneurs' option to delay entry, causing a "wait-and-see" friction. The author argues that if entrepreneurs can pause plans until aggregate
conditions improve, recessions will not only feature fewer entries but also many delayed projects, leading to a surge of entries at the subsequent recovery.

Finally, there are welfare and policy implications to consider. In models like BGM, there is a variety externality issue, where individual firms do not internalize
that their entry adds consumer surplus by increasing variety. This might lead to under-entry in equilibrium, suggesting a possible socially inefficient equilibrium 
and a role for subsidies to startup formation. However, other potential distortions such as markups, fixed costs, and financial frictions might complicate the picture, 
not making it certain that more entry is always welfare-adding. Recent research by \textcite{bilbiie2019monopoly} tackles this issue by deriving optimal monetary
and fiscal policy in these settings, finding that the planner typically wants to encourage firm entry, especially during downturns, to offset the tendency of recessions
to reduce varieties and future supply. \textcite{baqaee2020productivity} also discuss the possibility that frictions caused by
markup dispersions or any friction to resource reallocation can potentially cause within-sector misallocations, creating an allocative-efficiency externality.

The integration of firm heterogeneity and endogenous entry into macroeconomics has opened up many questions. The fundamental debate with which I occupy myself
in this thesis is: How important is firm heterogeneity for macroeconomic fluctuations? Although \textcite{bilbiie2012endogenous} show that 
pro-cyclical startup entry amplifies business-cycle volatility, they treat all entrants as technologically identical, while \textcite{sedlavcek2017growth} 
highlight how recessions skew the composition of entrants through matching frictions but do so outside a love-of-variety RBC setting and with a different question of
firm employment in mind. By embedding Sedláček–Sterk’s technology-choice and matching block inside the BGM framework, 
this thesis delivers a DSGE model in which sector-specific returns-to-scale heterogeneity, entry-stage matching frictions, and endogenous idea supply shocks jointly shape 
and enrich the classic RBC shock propagation mechanisms. The model delivers a more sophisticated picture of how aggregate shocks can affect different types of firms in the economy, and
how this heterogeneity in turn affects aggregate outcomes through rich firm dynamics not explored in the standard BGM framework.


%%%%%%%%%%%%%%%%%%%%%%%%%%%%%%%%%%%%%%%%%%%%%%%%%%%%%%%%%%%%%%%%%%%%%%%%%%%%%
%%%%%%%%%%%%%%%%%%%%%%%%%%%%%%%%%%%%%%%%%%%%%%%%%%%%%%%%%%%%%%%%%%%%%%%%%%%%%
%%%%%%%%%%%%%%%%%%%%%%%%%%%%%%%%%%%%%%%%%%%%%%%%%%%%%%%%%%%%%%%%%%%%%%%%%%%%%
\section{Model}
\label{sec:model}

In this section, I briefly lay out the firm side in the baseline BGM model with no capital and C.E.S.preferences in section \ref{sec:model-bilbiie}, explain the Sedláček–Sterk
heterogeneity and matching friction mechanisms in section \ref{sec:model-sedlacek}, and finally, fully characterize the main model and the departures made
from the standard BGM framework in section \ref{sec:model-mine}. 

%%%%%%%%%%%%%%%%%%%%%%%%%%%%%%%%%%%%%%%%%%%%%%%%%%%%%%%%%%%%%%%%%%%%%%%%%%%%%
%%%%%%%%%%%%%%%%%%%%%%%%%%%%%%%%%%%%%%%%%%%%%%%%%%%%%%%%%%%%%%%%%%%%%%%%%%%%%
\subsection{Baseline Bilbiie-Ghironi Framework (C.E.S. Preferences)}
\label{sec:model-bilbiie}

In the original BGM framework, the key elements of the firm side are fully characterized by the production/labor decision 
and an entry decision pinned down by a free entry condition in equilibrium. I do not fully explore the consumer side of the framework yet for the sake of brevity, since it is analogous to the consumer side in the main model which will be 
explained in detail later. As described before in section \ref{sec:literature}, each firm (or production line) produces a unique variety $\omega$ using labor as their single 
production factor, with the production function 
\[
    y_t(\omega) = Z_t \cdot l_t(\omega),
\]
where $Z_t$ is the exogenous aggregate productivity level and $l_t(\omega)$ is the firm's labor input. Consumers have C.E.S.preferences over a continuum
of varieties 
\[
    C_t = \left[\int_{\omega \in \Omega_t} c_t(\omega)^{\frac{\theta-1}{\theta}}\,d\omega\right]^{\frac{\theta}{\theta-1}},
\]

which implies a consumption-based price index and constant markups for the firm. Solving the firm's pricing problem arrives at the equilibrium real pricing
condition of

 \[
    \rho_t(\omega) = \frac{p_t(\omega)}{P_t} =  \mu \cdot \frac{w_t}{Z_t},
 \]

with $\mu=\frac{\theta}{\theta-1}$ and marginal costs corresponding to $\frac{w_t}{Z_t}$, which are symmetrical across firms/varieties.

In the entry block, firms face a sunk entry cost $f_{E,t}$, expressed in effective labor units, and once firms have successfully entered, they survive
with probability $(1-\delta)$, where $\delta \in (0,1)$ is a "death shock". The law of motion dictating the total number of producers is 

    \[
    N_t = (1-\delta)(N_{t-1} + N_{e,t-1}),
    \]
where $N_{e,t}$ is the measure of new entrants, which face a time-to-build lag before entering. In equilibrium, the free-entry condition equates the expected present value of a firm's
future profit stream to the sunk cost in units of the consumption good, such that entry occurs until firm value is equalized with entry cost:

\[
    v_t(w) = \frac{w_t f_{E,t}}{Z_t}.
\]

Prospective entrants compute their expected post-entry value $v_t(\omega)$ given the present discounted value of their expected stream of profits
$\{d_s(\omega)\}_{s=t+1}^\infty$: 
$$v_{t}(\omega) = E_t \sum_{s = t+1}^{\infty} \left[ \beta (1 - \delta) \right]^{s - t} \left( \frac{C_{s}}{C_{t}} \right)^{-1} d_{s}(\omega).$$

In the BGM framework, this also represents the value of incumbent firms after production has happened, which is relevant for the household's investment decision.



%%%%%%%%%%%%%%%%%%%%%%%%%%%%%%%%%%%%%%%%%%%%%%%%%%%%%%%%%%%%%%%%%%%%%%%%%%%%%
\subsection{The Sedláček–Sterk Heterogeneity Mechanism}
\label{sec:model-sedlacek}

I abstract away from almost all other relevant model aspects presented by \textcite{sedlavcek2017growth} 
in this section to focus on two key mechanisms: the technology/returns-to-scale choice and the entry-stage 
matching friction. Before entry, each prospective entrepreneur/startup chooses a technology type $i\in\{1,\dots,I\}$,
which pins down the constant-returns-to-scale parameter $\alpha_i\in(0,1)$. Once active, an incumbent of age $a$ 
produces according to 
\[
y_{i,a,t}=A_t\,n_{i,a,t}^{\alpha_i},
\]
where $A_t$ is aggregate TFP and $n_{i,a,t}$ denotes employment. The heterogeneity in $\alpha_i$ implies 
heterogeneous optimal sizes for different startups, with size referring to the size of employment $n_{i,a,t}$. Low-$\alpha_i$ firms remain small due to smaller scalability, whereas high-$\alpha_i$ firms can scale 
to thousands of workers over their life cycle, matching empirical cohort size distributions. 

After a business idea/technology type is chosen, the firms face the entry-stage matching friction. Let $e_{i,t}$ be
the measure of aspiring entrants that select technology $i$ at time $t$, and let $\psi_{i,t}$ denote the stock 
of business opportunities of type $i$ at time $t$ (with $\sum_i\psi_{i,t}=\Psi$). Because multiple entrants may chase the same idea, only
a fraction actually succeed. Successful startups $m_{i,t}$ are generated by a Cobb-Douglas matching function

\[
m_{i,0,t}=e_{i,t}^{\phi}\psi_{i,t}^{\,1-\phi},
\]
where $\phi\in(0,1)$ is the matching elasticity, such that the success probability is pinned down by:

\[
P_{i,t}\;=\;\frac{m_{i,0,t}}{e_{i,t}}=e_{i,t}^{\,\phi-1}\psi_{i,t}^{\,1-\phi}.
\]

After paying the sunk entry cost $\chi$, a potential entrant earns expected value
$P_{i,t}V_{i,0,t}(0,F_t)$, where $V_{i,0,t}$ is the value of a newborn firm at time $t$ for technology type $i$.
Free entry therefore requires
\[
\chi \;=\;P_{i,t}V_{i,0,t}(0,F_t)\qquad\forall i,  \tag{FE}
\]
which pins down $P_{i,t}$ (and hence $m_{i,0,t}$) endogenously. In their quantitative implementation, the authors treat the 
probability $P_{i,t}$ as a fixed parameter due to data availability issues with $\psi_{i,t}$, which is a strategy also adopted
by the main model presented later on.

The interaction between these two mechanisms creates rich firm dynamics. Because higher-$\alpha_i$ technologies yield 
larger $V_{i,0,t}$, condition (FE) forces their $P_{i,t}$ to fall—i.e.\ they face tougher matching—so shocks that 
shift the opportunity mix $\psi_{i,t}$ endogenously tilt the composition of entrants toward or away from high-growth firms. 
This joint mechanism of technology heterogeneity and matching frictions drives the persistent 
cohort‐quality effects documented before in section \ref{sec:literature}. 

%%%%%%%%%%%%%%%%%%%%%%%%%%%%%%%%%%%%%%%%%%%%%%%%%%%%%%%%%%%%%%%%%%%%%%%%%%%%%
%%%%%%%%%%%%%%%%%%%%%%%%%%%%%%%%%%%%%%%%%%%%%%%%%%%%%%%%%%%%%%%%%%%%%%%%%%%%%
\subsection{The Model}
\label{sec:model-mine}

The model is a variant of the RBC model with endogenous firm entry and monopolistic competition found in \textcite{bilbiie2012endogenous}, from which
I make two departures based on the heterogeneity mechanism found in \textcite{sedlavcek2017growth}: i) multiple sectors with heterogeneity in
returns-to-scale parameters $\alpha_i$; ii) matching frictions for aspiring entrants choosing and attempting to enter a sector.

I accommodate the technology choice mechanism by dividing the firm side of the economy into a finite number of sectors $I$ indexed by
$i \in \{1,2,...,I\}$. There is a continuum of unique firms/varieties indexed by $\omega \in \Omega$, with each one being fully defined by 
the unique product variety $\omega$ that it produces. I refer to firm $\omega$ that belongs to sector $i$ as "firm $\omega i$". 
I use $\Omega_i$ to denote the set that contains the indices of firms that belong to sector $i$, so that $\bigcup_{i=1}^{I} \Omega_i = \Omega$. 
Its measure, here denoted $M_{i}$, gives the size of the sector in terms of firms (or production lines). I adopt the same interpretation as in BGM 
of understanding each firm/variety as a production line rather than a separate entity, which also matches well with the scalability logic introduced
by a sectoral diversity in returns-to-scale parameters, as will be explored later in section \ref{sec:solution}.

\medskip
\medskip
\noindent\textbf{Households}
\medskip

The household framework is nearly identical to the one found in BGM, with the only difference being their preference specification. There is a 
unit mass of atomistic, identical households. All contracts are nominal. Prices are flexible and thus I only solve for real variables. As the
composition of the consumption basket changes due to firm entry, I introduce money as a unit of account only, such that it plays no other role in the
economy and no modeling of money demand is necessary, resorting to a cashless economy as in BGM and \textcite{woodford2003interest}.

The Household supplies $L_t$ in a competitive labor market for $W_t$, and it maximizes $E_t \left[ \sum_{s=t}^{\infty} \beta^{s - t} U(C_s, L_s) \right]$ 
where $C$ is consumption and $\beta \in (0,1)$ the subjective discount factor. The period utility function follows the form 
$U(C_t, L_t) = \ln C_t - \chi \frac{(L_t)^{1 + 1/\varphi}}{1 + 1/\varphi}, \quad \chi > 0$ where $\varphi \geq 0$ is the Frisch-elasticity
of labor supply to wages and the intertemporal elasticity of substitution in labor supply. This choice of functional form makes sure that 
income and substitution effects of real wage variation on effort cancel out in steady state, as argued in BGM.

At time t, the household consumes the basket of goods $C_t$, defined over a continuum of goods $\Omega$. At any time, only $\Omega_t \in \Omega$ 
is available. Given sector $i$, the continuum of goods stemming from that sector is $\Omega_{i}$, and the basket of goods is $C_{i,t}$. 
Let $p_{i,t}(\omega)$ denote the nominal price 
of a good $\omega \in \Omega_{i,t}$. Additionally, demand for an individual variety, $c_{i,t}(\omega)$, is obtained as 
$c_{i,t}(\omega) \, d\omega = C_{i,t} \, \partial P_{i,t} / \partial p_{i,t}(\omega)$.

As in BGM but adapted to the sectoral framework here presented, the symmetric price elasticity of demand in sector $i$ $\zeta_i$  is a function 
of the number $M_{i,t}$ of goods/producers such that $\zeta (M_{i,t}) = (\partial c_{i,t}(\omega)/\partial p_{i,t}(\omega))(p_{i,t}(\omega)/c_{i,t}(\omega))$, 
for any symmetric variety $\omega$. The benefit of additional product variety is here too described by the sectoral relative price $\rho_{i,t}(\omega) =
\rho(M_{i,t}) \equiv p_{i,t}/P_{i,t}$, for any symmetric variety $\omega$ in sector $i$, or in elasticity form $\epsilon(M_{i,t}) \equiv \rho'(M_{i,t})\,M_{i,t}/\rho(M_{i,t})$. 
Together, $\zeta(M_{i,t})$, and $\rho (M_{i,t})$ completely 
characterize the effects of consumption preferences in this model. I define explicit functional forms for these objects later when specifying the 
functional form of preferences within and across sectors below.


\medskip
\medskip
\noindent\textbf{Firms}
\medskip

The firm block is defined as a collection of $I$ heterogeneous sectors with $M_{i,t}$ homogeneous firms each, 
facing two decisions: In which sector to enter and labor demand for each period after entering. There is a continuum of monopolistically competitive
firms, each producing a different variety $\omega \in \Omega$, with one factor: Labor. 
Aggregate labor productivity is indexed by $Z_t$ (effectiveness of one unit of labor), which is 
exogenous, equal across sectors, and follows an $AR(1)$ process (in logarithms). For a firm that chooses sector $i$, the production function becomes
$y_{i,t}(\omega) = Z_t \cdot l_{i,t}(\omega)^{\alpha_i}$. Similar to the Sedláček–Sterk framework, firms chose a technology type/business idea $i$ 
that determines their returns-to-scale parameter $\alpha_i$ and thus directly influences their marginal product of labor.

Before entry, the firm pays $f_{E,t}$ effective labor units, equal to $\frac{w_tf_{E,t}}{Z_t}$ units of the consumption good. 
Firms that enter the economy produce every period until they get hit with a "death shock" (probability $\delta \in (0,1)$ in every period). 
Entrants at time t can only start producing at time t+1, such that they incur a one-period time-to-build lag. 
To produce a certain output level $y$, firms minimize their labor cost in a static minimization problem
\[
    \min_{l_{i,t}(\omega)} \; w_t\,l_{i,t}(\omega) \quad \text{subject to} \quad y = Z_t \cdot l_{i,t}^{\alpha_i}(\omega),
\]
such that the optimal labor demand is then given at time $t$ in sector $i$ by 
\begin{equation}
  l_{i,t}^*(\omega) = \left(\frac{y}{Z_t}\right)^{\frac{1}{\alpha_i}},\label{eq:laborClearing}
\end{equation}
and therefore marginal costs are expressed as
\begin{equation}
  \text{MC}_{i,t} = \frac{w_t}{\alpha_i Z_t}\left(\frac{y_{i,t}(\omega)}{Z_t}\right)^{\!1/\alpha_i-1},\label{eq:marginalcost}
\end{equation}
which, differently from BGM, are sector-specific and depend directly on the returns-to-scale parameter $\alpha_i$. However, the BGM
marginal cost expression is retrievable by setting $\alpha_i = 1, \, \forall i \in \{1,2,...,I\}$. With C.E.S. preferences, firms set relative
prices as constant markups over marginal costs 

\begin{equation}
  \rho_{i,t}(\omega) = \mu MC_{i,t}.\label{eq:pricing}
\end{equation}

As in BGM, but at the sectoral level, I assume a conditional symmetric firm equilibrium, where firms from the same sector 
are symmetric over prices, quantities, and values. 
This implies that $p_{i,t}(\omega) = p_{i,t}, \ \rho_{i,t}(\omega) = \rho_{i,t}   , \   l_{i,t}(\omega) = l_{i,t}   , 
\  y_{i,t}(\omega) = y_{i,t}   , \  d_{i,t}(\omega) = d_{i,t}   ,  \ v_{i,t}(\omega) = v_{i,t}   , \  
\frac{p_{i,t}}{P_{i,t}} = \rho_{i,t} = \rho(M_{i,t})$. Therefore, profits/dividends for all firms in 
sector $i$ are identical and given by
\begin{equation}
  d_{i,t} = \Bigl(1-\tfrac1\mu\Bigr)\frac{C_{i,t}}{M_{i,t}}, \label{eq:profit}
\end{equation}
where $y_{i,t} = \frac{C_{i,t}}{M_{i,t}}$ since all firms are symmetrical and supply equal amounts of demand in their sector, analogous to firm-level
output in the BGM framework but adapted for the sectoral framework.

\medskip
\medskip
\noindent\textbf{Firm Entry with Uncertainty}
\medskip

The matching friction mechanism found in \textcite{sedlavcek2017growth} is adapted to the RBC framework presented here, such that firms face matching 
frictions in entering a sector and not all potential entrants are successful. In every time period $t$ and sector $i$, $\exists M_{i,t}$ mass of 
producing product lines and unbounded mass of prospective entrants. The entrants are forward-looking and correctly anticipate their future 
profits $d_{i,s}$ in every period $s\ge t+1$ as well as $\delta$. Prospective entrants choose a technology type $i$ and the corresponding sector.
Let technology types reflect returns-to-scale in increasing order such that $i = 1$ is the lowest and $i = I$ is the highest. 
Let $e_{i,t}$ denote the measure of entrants choosing $i$ at time $t$. Not every attempt to enter is successful. The Cobb-Douglas  matching function 
is given by $m_{i,t}= e_{i,t}^{\phi}\,\psi_{i,t}^{\,1-\phi}$ such that $m_{i,t}$ are the successful entrants, $\psi_{i,t}$ is the measure of 
business opportunities for type $i$ at time $t$, and $\phi$ is a matching elasticity common across all types. 

In contrast with the Sedláček–Sterk formulation, in this new framework, the overall measure of business ideas \(\Psi\) is not necessarily held constant,
being affected over time due to an exogenous aggregate stock of ideas shock \(X_t\), but its composition is. The motivation behind this change is that
the original formulation investigates how the composition of cohorts influences employment, whereas this framework is interested in the aggregate 
effects of more business ideas being available, not a shift in their distribution. To stay consistent with this new formulation, I specify the measure 
of business opportunities for technology \(i\) as 
\begin{equation}
  \psi_{i,t} =X_t^{\frac{1}{1-\phi}}\,\bar{\psi}_i, \label{eq:psilaw}
\end{equation}
where \(\bar{\psi}_i\) is the steady-state measure of business opportunities for technology \(i\) and $X_t$ follows an AR(1) process of the type 
$X_t = 1 - \rho_X + \rho_X X_{t-1} + \epsilon^X_t$, where $\rho_X$ is a persistence parameter and $\epsilon^X_t$ are i.i.d innovations 
normally distributed with mean zero and variance $\sigma_X$. Consequently, the probability of a successful startup in sector $i$ is then:
\begin{equation}
  \Pi_{i,t} = \frac{m_{i,t}}{e_{i,t}} = e_{i,t}^{\phi-1}\,\psi_{i,t}^{\,1-\phi}. \label{eq:probability}
\end{equation}
Before entry, as in the BGM framework, prospective entrants compute their expected post-entry value by discounting the value of their expected stream
of profits $\{d_{i,s}\}_{s=t+1}^\infty$ as
\begin{equation}
  v_{i,t} = E_t \sum_{s = t+1}^{\infty} \left[ \beta (1 - \delta) \right]^{s - t} \left( \frac{C_{i,s}}{C_{i,t}} \right)^{-1} d_{i,s}, \label{eq:value}
\end{equation}
and thus entry occurs until firm value is equalized with entry cost, yielding the free entry condition for sector $i$ 
\begin{equation}
  \Pi_{i,t}v_{i,t} = f_{E,t}w_t/Z_t , \label{eq:freeentry}
\end{equation}
for all $i \in \{1,2,..., I\}$. This entry framework yields the new sectoral law of motion that dictates the number of incumbents in sector $i$, 
given by the following expression
\begin{equation}
  M_{i,t+1} = (1-\delta)(M_{i,t} + \Pi_{i,t}e_{i,t}). \label{eq:entrants}
\end{equation}
An important observation is that, as demonstrated in BGM, there is no option value for waiting, since there remains no 
idiosyncratic/firm-specific uncertainty and exit constitutes an exogenous process, meaning that uncertainty related to firm death is also aggregate. 
Despite the introduction of matching frictions, every potential entrant faces the same matching probabilities $\Pi_{i,t}$ for each sector $i$ 
determined by aggregate conditions that are not firm-dependent\footnote{See Section \ref{sec-app:option} in the appendix for the expanded BGM 
proof for my framework with matching frictions.}.

\medskip
\medskip
\noindent\textbf{Preference Specification and Markups}
\medskip

To capture heterogeneity across technology types, I adopt a two-layer aggregation process that nests the Dixit--Stiglitz framework at both 
the micro/sector and macro/aggregate levels, following \textcite{carvalho2021sectoral}. For each sector $i$, consumption is an aggregate of the
continuum of varieties \(\omega \in \Omega_i\). Specifically, the type-\(i\) consumption bundle is defined as
\[
C_{i,t} \;=\; \left[ \int_{\omega \in \Omega_i} c_{i,t}(\omega)^{\frac{\theta-1}{\theta}}\,d\omega \right]^{\frac{\theta}{\theta-1}},
\]
and the corresponding price index is given by
\[
P_{i,t} \;=\; \left[ \int_{\omega \in \Omega_i} p_{i,t}(\omega)^{\,1-\theta}\,d\omega \right]^{\frac{1}{1-\theta}},
\]
where \(\theta \ge 1\) is the elasticity of substitution among varieties within sectors. By construction, total expenditure on goods in 
sector \(i\) is \(P_{i,t}C_{i,t}\). At the aggregate level, the type-specific bundles \(\{C_{i,t}\}_{i=1}^I\) are 
aggregated into a final consumption basket. The final consumption aggregator is given by
\[
C_t \;=\; \left[ \sum_{i=1}^I \left( C_{i,t} \right)^{\frac{\eta-1}{\eta}} \right]^{\frac{\eta}{\eta-1}},
\]
with the overall price index defined as
\[
P_t \;=\; \left[ \sum_{i=1}^I \left( P_{i,t} \right)^{\,1-\eta} \right]^{\frac{1}{1-\eta}},
\]
where $\eta \ge 1$ is the elasticity across sectors. This top-level aggregation treats each technology type as a distinct variety, 
such that consumers can also substitute between sectoral bundles.

It follows that the household's demand for each good and sectoral good is 
$c_{i,t}(\omega) = \left(\frac{p_{i,t}}{P_{i,t}}\right)^{-\theta}C_{i,t}$ and 
$C_{i,t} = \left(\frac{P_{i,t}}{P_t}\right)^{-\eta}C_t$. Additionally, the markup and the within/across sector benefit of variety are 
independent of the number of goods, and related by $\epsilon = \mu -1 = 1/ (\theta-1)$. The C.E.S. expressions are then: 
$\mu(M_{i,t}) = \mu = \frac{\theta}{\theta - 1}, 
\   \epsilon_{\text{within}}(M_{i,t}) = \mu - 1 , \ \epsilon_{\text{across}}(N_{t}) = \eta - 1$ and most notably for the relative price
\begin{equation}
  \rho(M_{i,t}) = M_{i,t}^{\mu - 1} = \left(M_{i,t}\right)^{\frac{1}{\theta - 1}}. \label{eq:lovevariety}
\end{equation}

\medskip
\medskip
\noindent\textbf{Household Problem}
\medskip

As stated before in section \ref{sec:model-bilbiie}, the household problem here is analogous to the one found in BGM, since the consumers are
effectively "blind" to the sectoral diversity and consume a final aggregate basket of goods $C_t$. The only significant difference is regarding
the heterogeneity of the firms whose shares the household buys, but consumers only observe the aggregate quantities. Therefore, I define the aggregate
firm-level profit and value as 
$$
    d_{t} = \frac{\sum_iM_{i,t}d_{i,t}}{\sum_iM_{i,t}} \quad \text{and} \quad v_{t} = \frac{\sum_ie_{i,t}v_{i,t}}{\sum_ie_{i,t}},
$$
and $N_t$ as the measure of $\Omega_t$, representing the total number of incumbents in the economy at time $t$.

In this framework, there are 
2 types of assets that the household can buy in period $t$. The first are shares in a mutual fund for firms $x_t$ and the second are 
risk-free bonds $B_t$. Mutual funds pay a total profit in each period equal to the total profit of all firms that produce in $t$, namely $P_tN_td_t$. 
During $t$, the household buys $x_{t+1}$ shares in a mutual fund of $N_{H,t} = N_t + \mathcal{E}_{t}$ firms (those already operating at 
time $t$ and new potential entrants $\mathcal{E}_t = \sum_ie_{i,t}$), but only $N_{t+1} = (1-\delta)(N_t +N_{E,t})$ (where $N_{E,t} = 
\sum_im_{i,t}$) will produce and pay at time $t+1$, due to matching frictions and the death shock. The price of a claim to the 
future profit stream of the fund of $N_{H,t}$ firms at $t$ is equal to the nominal price of claims to future firm profits, $P_tv_t$.

Given this setup, the period budget constraint for the representative household can be written as
$$B_{t+1} + v_t N_{H,t} x_{t+1} + C_t = (1 + r_t) B_t + (d_t + v_t) N_t x_t + w_t L_t,$$
where the left side represents expenditures including the purchase of new assets and consumption, and the left side represents income
including the dividends from firms, returns on bond stocks, and labor income.

From this framework, I derive the first-order conditions of the household for optimal behavior by solving the utility maximization problem,
yielding the Euler equation for bonds 
\begin{equation}
(C_t)^{-1} = \beta (1 + r_{t+1}) E_t \left[ (C_{t+1})^{-1} \right], \label{eq:eulerbonds}
\end{equation}
the Euler equation for shareholdings in the fund of firms 
\begin{equation}
v_t = \beta (1 - \delta) E_t \left[ \left( \frac{C_{t+1}}{C_t} \right)^{-1} (v_{t+1} + d_{t+1}) \right], \label{eq:Eulershares}
\end{equation}
which, if iterated forward, yields the aggregate version of the relationship expressed by the firm valuation equation \eqref{eq:value}, and finally the intratemporal 
first-order condition 
\begin{equation}
\chi\,L_{t}^{1/\varphi}=\frac{w_t}{C_t}, \label{eq:intratemp}
\end{equation}
where $\varphi$ represents the Frisch-elasticity of labor and $\chi$ is the
disutility of labor, as before\footnote{As in BGM, I omit the transversality conditions for bonds and shares that must be satisfied to
ensure optimality. Additionally, the interest rate is determined residually since it appears only in the Euler equation for bonds and is pinned
down once consumption is known.}.

\medskip
\medskip
\noindent\textbf{Aggregate Accounting, Equilibrium, and the Labor Market}
\medskip

Under equilibrium, the following conditions hold for market clearing: $B_{t+1} = B_t=0$ and $x_{t+1}=x_t=1$ $\forall t$. Therefore, by applying these
equilibrium conditions to the household budget constraint and disaggregating firm-level variables, I arrive at the aggregate accounting identity
\begin{equation}
C_t + v_t\mathcal{E}_t= w_t.L_t + d_tN_t \;\Leftrightarrow\;C_t + \sum_iv_{i,t}e_{i,t}= w_t.L_t + \sum_id_{i,t}M_{i,t}, \label{eq:aggaccount}
\end{equation}
which states that the expenditure in consumption and investment in new firms must be equal to the income derived from labor and 
firm dividends in equilibrium.

Similar to the dynamic in the BGM baseline model, and in contrast to the standard one-sector RBC model of \textcite{campbell1994inspecting}, 
the model presented here has two "macro-sectors", such that one employs part of the labor supply to produce consumption goods and the other
sector employs the rest of the labor supply to establish new production-lines/firms. Consequently, the economy's GDP $Y_t$ is simultaneously
equal to total income $w_t.L_t + \sum_id_{i,t}M_{i,t}$ and the total output of the economy $C_t + \sum_iv_{i,t}e_{i,t}$, and $v_{i,t}$ represents 
the relative price of investment in terms of consumption.

As a direct consequence of the two-sector nature of the model, the labor market equilibrium is characterized by $L^C_t+L^E_t = L_t$, that is, 
the total labor is equal to the labor used in production $L^C_t = \sum_iM_{i,t}\ell_{i,t}$ at time $t$ plus the total labor 
used to build firms $L^E_t = \sum_ie_{i,t} f_{E,t}/Z_t$ at time $t$. Entry determines the amount of labor allocated to setting 
up new production lines, such that entry at time $t$ affects labor demand in time $t+1$. Therefore, aggregate labor clearing can be expressed as
\begin{equation}
 L_t = \underbrace{\sum_i M_{i,t}l_{i,t}}_{\text{production}} + \underbrace{\sum_i e_{i,t}f_{E,t}/Z_t}_{\text{entry}}.\label{eq:AgglaborClearing}
\end{equation}
Matching frictions, however, introduce a dynamic where not all the labor spent in building new firms is fruitful. 
Labor is used to produce firms that might not survive the matching
process, especially for highly scalable sectors that have lower probabilities of successful entry.

\medskip
\medskip
\noindent\textbf{Characterizing Equilibrium}
\medskip

Given predetermined variables $M_{i,t}$, $\psi_{i,t}$, $r_t$ and exogenous shocks $(Z_{t},X_{t}, f_{E,t})$\footnote{Note: I treat 
$f_{E,t}$ as a parameter during quantitative exercises.}, the intra-period sequence in the model goes as follows: i) Given their preferences, households
choose $\{C_t,L_t,B_{t+1},x_{t+1}\}$; ii) Incumbent firms hire labor, set prices, produce and pay dividends in all sectors;
iii) Potential entrants pay the sunk entry cost $f_{E,t}w_t/Z_t$.  Matching occurs with probability $\Pi_{i,t}$ and 
successful entrants $m_{i,t}$ join $M_{i,t+1}$ in all sectors $i$ (time-to-build); iv) Exogenous shocks $(Z_{t+1},X_{t+1})$ realize and 
date $t{+}1$ begins.

Out of the 15 main equations highlighted above, many of the variables introduced are merely definitions and/or identities that add no new 
information to the system. 
For instance, the definition of firm-level output and labor demand \eqref{eq:laborClearing} simply build $y_{i,t}$ and $l_{i,t}$ 
once \( C_{i,t} \), \( M_{i,t} \) are known. Additionally, the aggregate consumption index $C_t$ is not included since it is mechanically 
defined once $C_{i,t}$ is known for each sector, adding no information to the system. Therefore, I do not include 
the aggregate goods clearing condition as an equilibrium condition, since it does not pin down any additional variable.
Furthermore, nominal variables like $P_{i,t}$ and $p_{i,t}$ are not solved for, since, with C.E.S. preferences, blocks 
\eqref{eq:pricing} and \eqref{eq:lovevariety} solve for the real/relative sectoral price $\rho_{i,t}$


Therefore, equilibrium can be characterized as follows: Given exogenous processes $(Z_t,X_t,f_{E,t})$, initial masses $\{M_{i,0}\}$, 
and a top-level price numéraire $P_t \equiv 1$ as a nominal anchor, I summarize equilibrium as a sequence 
\[
\bigl\{\,w_t^*,L_t^*,r_{t}^*,\bigl\{
        M_{i,t}^*,C_{i,t}^*,\rho_{i,t}^*,d_{i,t}^*,v_{i,t}^*,e_{i,t}^*,\psi_{i,t}^*,\Pi_{i,t}^*\bigr\}^I_{(i=1,\dots,I)}\bigr\}.
\]
of 3 + 8$I$ endogenous variables, with 3 variables being predetermined at time $t$ ($M_{i,t}$, $r_t$ and $\psi_{i,t}$), 
satisfying the simplified system of equilibrium conditions in Table \ref{tb:eqconditions} out of all the relevant equations explored in section \ref{sec:model-mine}

\begin{table}[H]
\centering
\renewcommand{\arraystretch}{1.4}
\begin{tabularx}{\textwidth}{@{}>{\raggedright\arraybackslash}p{0.4cm} >{\raggedright\arraybackslash}p{3.5cm} X c@{}}
\toprule
\textbf{} & \textbf{Type} & \textbf{Equation(s) in Section 3} & \textbf{Dimension} \\
\midrule
1 & Pricing Rule in sector $i$ & \(  \rho_{i,t} = \mu \, \frac{w_t}{\alpha_i Z_t}\left(\frac{y_{i,t}}{Z_t}\right)^{\!1/\alpha_i-1}  \quad \text{combining \eqref{eq:pricing} and \eqref{eq:marginalcost}} \) & \( I \) \\
2 & Variety Effect in sector $i$ & \(  \rho_{i,t} = M_{i,t}^{\frac{1}{\theta-1}}  \quad \text{ from \eqref{eq:lovevariety}} \) & \( I \) \\
3 & Profits in sector $i$ & \(d_{i,t} = \Bigl(1-\tfrac1\mu\Bigr)\frac{C_{i,t}}{M_{i,t}} \quad \text{ from \eqref{eq:profit}}\) & \( I \) \\
4 & Firm valuation in sector $i$ & \(v_{i,t}=\beta(1-\delta)E_t\Bigl[\bigl(C_{i,t+1}/C_{i,t}\bigr)^{-1}\bigl(v_{i,t+1}+d_{i,t+1}\bigr)\Bigr]\quad \text{from \eqref{eq:value}}\) & \( I \) \\
5 & Free entry in sector $i$ & \(  \Pi_{i,t}=\frac{f_{E,t}\,w_t}{Z_tv_{i,t}}  \quad \text{ from \eqref{eq:freeentry}} \) & \( I \) \\
6 & Stock of ideas in sector $i$ & \(  \psi_{i,t}=X_t^{\frac{1}{1-\phi}}\bar{\psi}_i  \quad \text{ from \eqref{eq:psilaw}} \) & \( I \) \\
7 & Matching definition & \( e_{i,t} = \psi_{i,t}\bigl(\Pi_{i,t}\bigr)^{\frac{1}{\phi-1}} \quad \text{from \eqref{eq:probability}} \) & \( I \) \\
8 & Number of firms in sector $i$ & \( M_{i,t+1} = (1-\delta)\bigl(M_{i,t}+e_{i,t}\Pi_{i,t}\bigr) \quad \text{from \eqref{eq:entrants}} \) & \( I \) \\
9 & Intra-temporal labor FOC & \( \chi L_t^{1/\phi} = \frac{w_t}{C_t} \quad \text{from \eqref{eq:intratemp}} \) & \( 1 \) \\
10 & Inter-temporal Euler (bonds) & \(1 = \beta(1+r_{t+1})E_t\Bigl[\tfrac{C_t}{C_{t+1}}\Bigr] \quad \text{from \eqref{eq:eulerbonds}} \) & \( 1 \) \\
11 & Aggregate Labor Identity & \( L_t=\sum_i(M_{i,t}\ell_{i,t}+e_{i,t} f_{E,t}/Z_t) \ \  \ \text{from \eqref{eq:AgglaborClearing}}\) & \( 1 \) \\
\midrule
\multicolumn{3}{r}{\textbf{Total independent conditions}} & \( 3 + 8I \) \\
\bottomrule
\end{tabularx}
\caption{Equilibrium Conditions and Dimensionality}
\label{tb:eqconditions}
\end{table}

The household Euler equation for shares is left out of the list of equilibrium conditions as in BGM because it represents the same relationship 
as the firm valuation equation but at an "average" aggregate level. Since I am interested in the sectoral profit and firm value dynamics, I omit it from the 
equilibrium conditions. Finally, when building the "$3 + 8I$" system, I keep only three unknowns per sector for the 
matching-entry block, $ e_{i,t},  \psi_{i,t}$ and $\Pi_{i,t}$. The auxiliary object \( m_{i,t} \) is left out of the list, 
precisely because it is mechanically pinned down once \( (M_{i,t}, e_{i,t}, \psi_{i,t},\Pi_{i,t}) \) are known.


%%%%%%%%%%%%%%%%%%%%%%%%%%%%%%%%%%%%%%%%%%%%%%%%%%%%%%%%%%%%%%%%%%%%%%%%%%%%%
%%%%%%%%%%%%%%%%%%%%%%%%%%%%%%%%%%%%%%%%%%%%%%%%%%%%%%%%%%%%%%%%%%%%%%%%%%%%%
%%%%%%%%%%%%%%%%%%%%%%%%%%%%%%%%%%%%%%%%%%%%%%%%%%%%%%%%%%%%%%%%%%%%%%%%%%%%%
\section{Model Solution and Properties}
\label{sec:solution}


%%%%%%%%%%%%%%%%%%%%%%%%%%%%%%%%%%%%%%%%%%%%%%%%%%%%%%%%%%%%%%%%%%%%%%%%%%%%%
%%%%%%%%%%%%%%%%%%%%%%%%%%%%%%%%%%%%%%%%%%%%%%%%%%%%%%%%%%%%%%%%%%%%%%%%%%%%%
\subsection{Sector–by–Sector Steady State}
\label{sec:solution-steadystate}

To derive the steady-state expressions for all variables, I divide the solution into $I$ sectoral blocks and one aggregate block, to stay consistent
with the formulation of equilibrium in section \ref{sec:model-mine}. I normalize aggregate productivity $Z = 1$ and set the stock of ideas shock
$X = 1$ in steady state, but I keep the variables in the expressions for tractability. Every variable is constant over time and thus time subscripts
are dropped. Otherwise, the notation matches the main model description in section \ref{sec:model-mine}.

From the intertemporal Euler equation for bonds, the steady-state interest rate is already pinned down by $1+r = \beta^{-1}$. I follow 
BGM and \textcite{campbell1994inspecting} and make use of $1+r = \beta^{-1}$ to treat $r$ as a parameter in the solution. The firm valuation
equation \eqref{eq:value} together with the definition $\kappa=\frac{r+\delta}{1-\delta}$ yields $v_i = d_i/\kappa$, which captures a premium for
the expected firm destruction, since a higher $\delta$ brings down the price of the investment "good" (\textit{ceteris paribus}). Combining 
this expression with the free-entry condition \eqref{eq:freeentry} and substituting in the definition of profits \eqref{eq:profit}, I arrive at  
\begin{equation}
  \frac{C_i}{M_i}=\frac{\kappa\,w f_E}{Z\Pi_i(1-1/\mu)}, \label{eq:CoverM}
\end{equation}
which pins down the ratio between sectoral consumption and the number of firms as a function of model parameters and the probability of successful 
entry in sector $i$, and also represents the definition of firm-level output $y_i$. This expression helps pin down in steady state all 
variables dependent on $y_i$ such as $\ell_i$ with equation \eqref{eq:laborClearing}, $MC_i$ with equation \eqref{eq:marginalcost} and 
$\rho_i$ with \eqref{eq:pricing}. 

Expression \eqref{eq:CoverM} hints at the fundamental role that competition plays in pinning down firm-level
behavior in the conditional symmetric firm equilibrium. Almost counterintuitively, an increase in the death rate, real wage, or sunk entry-cost all lead
to an increase in the firm-level output and labor demand, given that all these changes decrease the incentives for entering sector $i$. This, in turn,
means that active incumbents in sector $i$ have fewer competing firms with whom to share the market, leading to a higher level of output and larger
firm size despite the negative impact of an increase in wages. 

The contrasting behavior of potential entrants and active incumbents comes from the fact that, after the forward-looking choice of entry, 
successful entrants only perform the myopic-static choice of optimal labor demand every period, and do not take into consideration their probability
of exit in doing so. Additionally, the negative impact of a wage surge on the labor decision is counterbalanced by the 
positive impact of a decrease in incumbents due to rising entry costs on within-sector competition, hence the positive impact of wage on firm-level output 
and employment.

Finally, an increase in the probability of successful matching also drives up entry, since an increase in $\Pi_i$ raises the expected payoff of attempting
an entry in sector $i$. In steady state, wage rises one-to-one with productivity since the marginal product of labor (MPL) is 
$MPL_i = \alpha_i Z l_i^{\alpha_i -1}$, such that only the ratio $w/Z$ actually matters, and thus a momentary increase in $Z$ has no direct impact
on firm-level output in a steady-state static equilibrium. Markups also have an interesting impact on firm-level output and size, where an increase in 
markups drives firm-level profits-per-unit up which makes entry appear more attractive to prospective entrepreneurs, raising $M_i$.
On top of that, however, higher markups also drive household demand for goods down since they increase relative prices, 
lowering $C_i$. These two effects combined make rising markups lower firm-level output.

As for the sectoral entry-matching block, combining the law-of-motion of incumbents \eqref{eq:entrants} and the matching 
definition \eqref{eq:probability} both in steady state yields 
\begin{equation}
m_i= e_i\Pi_i= \frac{\delta}{1-\delta}M_i. \label{steadyentrants} 
\end{equation}
That is, the number of successful new entrants $m_i$ makes up for the exogenous destruction of new firms, and the 
number of prospective new entrants $e_i$ also compensates for the probability $\Pi_i$ in each sector of successful
matching taking place.

The aggregate accounting identity \eqref{eq:aggaccount} in steady state is also particularly informative when it
comes to understanding the general structure of the model. By substituting in the free entry condition 
\eqref{eq:freeentry}, matching probability definition \eqref{eq:probability}, and the law of motion of incumbents \eqref{eq:entrants}, 
all in steady state, the identity now becomes:
\begin{equation}
C + \frac{w f_E}{Z}\frac{\delta}{1-\delta}\sum_iM_{i}= w.L + \frac{w f_E}{Z}\kappa \sum_i\frac{M_i}{\Pi_i}. \label{aggsteadystate} 
\end{equation}
The same general idea of matching expenditure (consumption plus investment) with income (labor income plus dividends)
in a two "macro-sector" economy remains. However, investment and dividends can be broken down into further detail. 

Investment, on the one hand, now captures the replacement principle seen in the classic RBC model 
(See \cite{campbell1994inspecting}) more clearly, where investment is made at steady state as pure maintenance to keep the variety/firm stock from shrinking. Therefore,
a higher death rate immediately requires more labor to be spent on producing new firms. Dividends, on the other hand, represent in
steady state the profits required to finance this replacement. A higher death rate, through $\kappa$, makes households require 
higher payoffs to keep financing new entries. Better matching in the form of a higher $\Pi_i$ also frees up profits that can go to 
consumption or lower labor effort, an effect that is masked in the non-static version \eqref{eq:aggaccount} because $e_{i,t}$ 
simultaneously adjusts on the investment side.

The blocks are,
as discussed previously, divided into sectoral and aggregate. Due to concerns introduced by the heterogeneous sectoral blocks, no
one-size-fits-all closed-form steady-state solution can be found, mainly due to the real wage. For that matter, I solve for the
steady-state values of all variables using an outer-inner loop dynamic explored in the following section \ref{sec:solution-num}.
Subsequently, I explore these results in section \ref{sec:solution-properties}, with a particular focus on firm dynamics and the
composition of the firm sector.


%%%%%%%%%%%%%%%%%%%%%%%%%%%%%%%%%%%%%%%%%%%%%%%%%%%%%%%%%%%%%%%%%%%%%%%%%%%%%
%%%%%%%%%%%%%%%%%%%%%%%%%%%%%%%%%%%%%%%%%%%%%%%%%%%%%%%%%%%%%%%%%%%%%%%%%%%%%
\subsection{Numerically Solving the Steady State}
\label{sec:solution-num}

In practice, the steady state routine is a two‐tier guess‐and‐verify scheme. 
First, I propose a trial real wage \(w^{(0)}\) that the household and firms would face. 
Given this provisional wage, each sector solves its conditional problem, pinning down \(M_i\) initially, after which $e_i$
and \(C_i\) follow algebraically, yielding firm-level output, labor, profits, and value. 
These quantities are then aggregated to compute total consumption \(C(w^{(0)})\) and labor \(L(w^{(0)})\), allowing the
evaluation of the household FOC residual
\begin{equation}
g\bigl(w^{(0)}\bigr) \;=\; \chi\,L(w^{(0)})^{1/\varphi} \;-\; \frac{w^{(0)}}{C\bigl(w^{(0)}\bigr)}. \label{bisection}
\end{equation}

The routine then updates the wage guess until \(g(w)=0\), at which point, the last sectoral solutions constitute 
the general equilibrium steady-state values for the sectoral variables, which are then aggregated to compute total quantities.
The exogenous inputs used to pin down the steady-state quantities are summarized in Table \ref{tb:exoinputs}. They are broken
down based on the block of the model where they have a direct impact. As mentioned in section \ref{sec:model-sedlacek}, 
there is no empirical counterpart to the stock of ideas variable $\psi_i$, only to the matching probability $\Pi_i$.
Therefore, as in \textcite{sedlavcek2017growth}, I take backed-out steady-state values for $\Pi_i$ from BDS firm data 
in the steady-state calculations performed here. The exact calibration used in steady-state calculations and the 
quantitative exercise is discussed in depth in section \ref{sec:quant-cal}.

\begin{table}[H]
\centering
\begin{tabular}{@{}ll@{}}
\toprule
\textbf{Block} & \textbf{Parameters} \\ \midrule
Preferences & $\,\beta,\,\chi,\,\varphi\quad\bigl(r=\beta^{-1}-1,\;
                 \kappa=\dfrac{\,r+\delta\,}{1-\delta}\bigr)$ \\[4pt]
Technology  & $\,\theta>1,\;\eta>1,\;Z\;(=1),\;
               \{\,\alpha_i\}_{i=1}^{I},\; \{\,\Pi_i\}_{i=1}^{I}$ \\[4pt]
Entry / matching & $\,f_E,\;\delta,\;\phi,\;X\;(=1)$ \\[4pt]
Markup constant & $\,\displaystyle
                   \mu=\frac{\theta}{\theta-1}$ \\ \bottomrule
\end{tabular}
\caption{Exogenous inputs for the steady-state routine}
\label{tb:exoinputs}
\end{table}

 
The main variables that I am interested in are the 3 + 8$I$ endogenous variables from equilibrium, but the routine also
pins down the whole system including the other variable definitions. Every sectoral variable is expressed as a function of 
the single scalar $w$, the current real wage guess. By equating the cost-based relative price \eqref{eq:pricing} with the 
love-for-variety expression \eqref{eq:lovevariety} 
and plugging in \eqref{eq:CoverM}, the result is a single scalar equation for $M_i$ as a function of the model parameters,
the real wage, and the matching probability:
\begin{equation}
  M_i^{\frac1{\theta-1}}
=\mu\,\frac{w}{Z\alpha_i}
\Bigl[\frac{\kappa\,w\,f_E}{Z(1-1/\mu)\,\Pi_i^*}\Bigr]^{\frac1{\alpha_i}-1}, \label{Mscalar}
\end{equation}

Solving \eqref{Mscalar} numerically given the current wage guess yields the
steady-state number of varieties/firms for each sector. After this step, I pin down the sectoral consumption level $C_i$ using 
\eqref{eq:CoverM} (and thus firm-level output $y_i$ and employment $l_i$ are also mechanically defined), the relative 
price $\rho_i$ using \eqref{eq:lovevariety}, firm-level profits $d_i$ using \eqref{eq:profit}, firm value $v_i$ using 
$v_i = d_i/\kappa$ and the number of prospective entrants $e_i$ using \eqref{steadyentrants} (which pins down both $m_i$
and $\psi_i$ by the matching probability definition \eqref{eq:probability}).

Subsequently, the aggregate final quantities such as $C$ and $L$ are mechanically determined. Consequently, the outer loop
using the intratemporal household FOC can be closed by finding the unique $w^*$ such that $g(w^*)= 0$. Since \eqref{bisection} is a 
one-dimensional monotone function (when $w$ rises, labor demand $L(w)$ falls and consumption $C(w)$ rises, 
hence $g(w)$ is strictly decreasing), a unique root exists, and a bisection algorithm converges fast. At convergence, the set
\[
\bigl\{\,w^*,L^*,r^*,\bigl\{
        C_i^*,M_i^*,\rho_i^*,d_i^*,v_i^*,e_i^*, \psi_i^*,\Pi_i^*\bigr\}^I_{(i=1,\dots,I)}\bigr\},
\]
satisfies all equilibrium conditions listed in Table \ref{tb:eqconditions}, with all other variables being 
pinned from their definitions and the numéraire being $P \equiv 1$.


%%%%%%%%%%%%%%%%%%%%%%%%%%%%%%%%%%%%%%%%%%%%%%%%%%%%%%%%%%%%%%%%%%%%%%%%%%%%%
%%%%%%%%%%%%%%%%%%%%%%%%%%%%%%%%%%%%%%%%%%%%%%%%%%%%%%%%%%%%%%%%%%%%%%%%%%%%%
\subsection{Properties of the Steady State}
\label{sec:solution-properties}

In this section, I explore the resulting dynamics of the numeric steady-state solution achieved through the 
algorithm in the previous section, with a special focus on the differences in behavior between different technology types.
Figure \ref{fig:steady-output-log} shows three important dimensions for each sector, namely firm-level output, firm-level 
employment, and the number 
of varieties/firms in each sector, which is represented by the size of each circle both in the figure and the legend.
Focusing first on the relationship between firm labor demand and output, the axes are logged to make the power law governing the relationship 
clearer, especially how the returns-to-scale coefficient affects this relationship. That is, by taking logs of equation 
\eqref{eq:laborClearing}, the relationship between labor and output at the firm-level becomes
\begin{equation}
  \ln l_i = \frac{1}{\alpha_i}(\ln y_i - \ln Z), \label{eq:laboroutputlog}
\end{equation}
such that the slope in the figure is $1/\alpha_i$, capturing the effects of different returns-to-scale parameters on the labor requirement
per unit of output. 

\begin{figure}[H]
  \centering
  \includegraphics[width=1\textwidth]{../images/steady_state/elastic_labor/steady_firm_dynamic_log.pdf}
  \caption{Steady State relationship between firm-level output and employment (logged).}
  \label{fig:steady-output-log}
\end{figure}

The different slopes are particularly clear when comparing the position of the circles in relation to the 45-degree dashed line. 
Given the small differences in the $\alpha$ parameters gathered from BDS data, differences in slope are small ($1/\alpha_i$ ranges 
only from 1.12 to 1.01), but not imperceptible. The higher $\alpha_i$, the closer the circle's center is to the 45-degree line, since 
for $\alpha_i = 1$ and no change in productivity $Z$, labor demand $l_i$ moves one-to-one with $y_i$ as evident by \eqref{eq:laboroutputlog}.

Therefore, the further a sector's bubble is from the 45-degree line, the smaller their returns-to-scale are, and as a consequence, output
rises noticeably less than employment, so the firm quickly runs into diminishing returns and stays small (low employment). On the other hand, 
for high-scalability sectors, a one-percent increase in employment raises output by almost one percent. Because marginal product falls slowly,
it takes longer for the firm to face diminishing returns to employment, and thus the firms' optimal size/employment is larger.

A real-life analogy could be the difference between a multi-plant manufacturer (highly scalable) and a specialized B2B producer 
(medium-small scalability). For the multi-plant manufacturer, despite their large fixed cost base, their standardized processes, efficient IT, 
or big capital stock lets each extra worker generate almost proportional extra output. On the other hand, for a specialized B2B producer output
is tied closely to a small specialized crew, and extra workers might not contribute with equal extra output due to coordination/training costs.

An interesting case is sector 9, which significantly stands above the rest in both output and employment levels. This is due to its substantial
returns-to-scale parameter, especially when compared to the preceding sectors 8 and 7. The increase in $\alpha$ from sector 7 to 8 
is merely $0.004$, whereas from sector 8 to 9 it is three times bigger at $0.012$. This significantly impact the size to which firms in 
sector 9 can grow when compared to other sectors, leading to bigger employment and output levels.

Moving on to the number of firms/varieties $M_i$, it is important to recall that, in steady-state, the higher the returns-to-scale parameter
in a sector, the smaller the matching probability $\Pi_i$ due to a limited stock of ideas and excessive competition. Therefore, one might find
the high number of firms in sectors 9, 8, or 7 contradictory, since entry is harder and firms are larger and thus the number of incumbents should
be smaller. This would, however, be an incorrect interpretation, since $M_i$ should be interpreted as the number of product lines (or varieties) 
rather than the number of stand-alone legal entities, as this is the interpretation favored in the baseline BGM model without 
sectoral heterogeneity. 

In this reading, each incumbent variety corresponds to a point on the continuum $\omega$, and the results displayed in 
Figure \ref{fig:steady-output-log} can be made sense of. Highly scalable firms end up having a bigger number of product lines due to the
easiness of scaling up the business, whereas low-scalability industries cannot expand their varieties/product-lines so easily due 
to scaling issues. Therefore, for high-$\alpha$ technologies, marginal labor requirements to generate output are low, so once a 
core platform is in place (such as Amazon's fulfillment network or a cloud software data center), adding another feature or product line costs
only the fixed entry cost. By contrast, labor-intensive sectors such as restaurants, bespoke services, and others, have lower returns to scale. Even with abundant ideas,
they cannot scale far, so their steady-state variety counts remain small due to the small number of prospective entrants.

However, starting a successful product line in a high-$\alpha$ business is inherently harder, due to the high competition in entry and the 
limited stock of ideas. This is balanced out by the high expected profits that translate into a high net present value of $v_i$,
making the free-entry condition \eqref{eq:freeentry} accommodate a vast flow of attempts, which in equilibrium yields a large stock
of active varieties $M_i$ (that might be all owned by the same firm, but the model abstracts away from this perspective). However, the difficulty of
entry starts to overpower the high scalability factor at some point. This is particularly true for sector 9, whose number of active varieties $M_9$ is slightly
smaller than $M_8$ for sector 8, despite the significantly higher value of $\alpha_9$, due to the extremely low probability of successful entry in 
steady-state $\Pi_9$ of around $0.2\%$\footnote{For reference, $\Pi_8$ is $1.3\%$ and $\Pi_7$ is $2.2\%$.}.

Another interesting aspect of the steady-state solution is the friction of entry and the resulting relationship 
between prospective entrants and successful ones, depicted in Figure \ref{fig:steady-entry-log}. 
The horizontal axis represents the steady-state number of prospective
entrants in sector $i$, $e_i$, and the vertical axis the steady-state number of successful entrants in sector $i$, $m_i$. The $\alpha_i$
parameter for each sector is indicated next to each circle, and the steady-state matching probability $\Pi_i$ for each sector is shown 
in the legend.

\begin{figure}[H]
  \centering
  \includegraphics[width=1\textwidth]{../images/steady_state/elastic_labor/steady_entry_log.pdf}
  \caption{Steady State relationship between prospective and successful entrants (logged).}
  \label{fig:steady-entry-log}
\end{figure}
As before, the axes are logged to reveal the elasticity between variables.
Therefore, taking logs turns the relationship into a line whose slope is elasticity $\epsilon_i = \frac{\Delta \ln m_i}{\Delta \ln e_i}$. 
The figure puts the matching friction aspect of the model into the spotlight, showcasing the "decresing-returns-to-entry" 
as one moves up the $\alpha_i$ values (moves down the matching probabilities $\Pi_i$). That is, on the double-log axes the 
curvature shows that the elasticity between potential entrants $e_i$ and successful startups $m_i$ falls steadily as the entry pool 
gets larger. In other words, each successive sector is fighting steeper diminishing returns-to-entry than the one to its left. 

This concurs with the logic found in the previous section, where high-$\alpha$ sectors are more attractive despite the lower probability
of success and low-$\alpha$ sectors face little congestion due to their unattractiveness, despite the large stock of ideas.
By sector 9, the success rate is so minimal that the curve starts to invert on itself, such that, as explored before for 
Figure \ref{fig:steady-output-log}, the low-matching-probability effect starts to dominate the high-net-present-value effect, beginning
to showcase a "Laffer-like" (\cite{laffer1974balance}) relationship between $e_i$ and $m_i$.

Interpreting the flow of prospective entrants $e_i$ as investment outlay, another insight about Figure \ref{fig:steady-entry-log} appears.
Namely, that it can be re-read almost one-for-one as a sector-specific capital-production function $m_i = F_i(e_i)$.
Therefore, the slope now represents the elasticity of new capital concerning investment. Because only a fraction $\Pi_i$ is 
successful, the gap $e_i - m_i$ is akin to depreciation in transit, and the curved shape pictures the declining marginal 
product of piling ever more entry resources into a sector where profitable ideas are increasingly scarce. In sum, the introduction of
matching frictions also enriches the "firm-entry-as-investment" dynamics of the original BGM framework, creating a more elaborate
process of capital creation where not all investment translates directly to  capital/firm formation.


%%%%%%%%%%%%%%%%%%%%%%%%%%%%%%%%%%%%%%%%%%%%%%%%%%%%%%%%%%%%%%%%%%%%%%%%%%%%%
%%%%%%%%%%%%%%%%%%%%%%%%%%%%%%%%%%%%%%%%%%%%%%%%%%%%%%%%%%%%%%%%%%%%%%%%%%%%%
%%%%%%%%%%%%%%%%%%%%%%%%%%%%%%%%%%%%%%%%%%%%%%%%%%%%%%%%%%%%%%%%%%%%%%%%%%%%%
\section{Quantitative Implementation}
\label{sec:quant}

In this section, I will focus on exploring the mechanisms of shock propagation in the model using a quantitative exercise.
I compute the impulse responses to a productivity shock analogous to the one found in BGM, and to the supply-of-ideas shock which is,
as explored in section \ref{sec:model-mine}, based on the composition shock in the Sedláček-Sterk quantitative exercise. 
The results corroborate the intuition presented in the previous sections while adding novel insights about the model. In line with the 
BGM approach, I present the impulse response functions (IRFs) only for the inelastic labor case, with the IRFs for the elastic labor
being found in the appendix in section \ref{sec-app:full}.

%%%%%%%%%%%%%%%%%%%%%%%%%%%%%%%%%%%%%%%%%%%%%%%%%%%%%%%%%%%%%%%%%%%%%%%%%%%%%
%%%%%%%%%%%%%%%%%%%%%%%%%%%%%%%%%%%%%%%%%%%%%%%%%%%%%%%%%%%%%%%%%%%%%%%%%%%%%
\subsection{Calibration}
\label{sec:quant-cal} 

In my baseline calibration, I adopt parameters from both \textcite{bilbiie2012endogenous} and \textcite{sedlavcek2017growth} and
interpret periods as years. Therefore, I adapt quarterly parameters from BGM, such as the exogenous firm exit shock $\delta = 0.09631$, 
to a yearly basis. All calibration values and their targets can be found in Table \ref{tab:calib_params_rotated}.

\begin{sidewaystable}[htbp]
\centering
\caption{Calibrated parameters}
\label{tab:calib_params_rotated}
\begin{tabular}{lll}
\toprule
Parameter & Value & Target \\
\midrule
\begin{tabular}[t]{@{}l@{}}$\beta$\\discount factor to match the 4\% annual interest rate\end{tabular}
  & 0.96
  & \textcite{sedlavcek2017growth} \\

\begin{tabular}[t]{@{}l@{}}$\delta$\\exogenous firm exit shock (yearly)\end{tabular}
  & 0.09631
  & \textcite{bilbiie2012endogenous} \\

\begin{tabular}[t]{@{}l@{}}$\theta$\\C.E.S. within-sector parameter\end{tabular}
  & 3.8
  & \textcite{bilbiie2012endogenous} \\

\begin{tabular}[t]{@{}l@{}}$\eta$\\C.E.S. across-sector parameter\end{tabular}
  & 2
  & \textcite{carvalho2021sectoral} \\

\begin{tabular}[t]{@{}l@{}}$f_{E}$\\steady‐state entry cost\end{tabular}
  & 1
  & \textcite{bilbiie2012endogenous} \\

\begin{tabular}[t]{@{}l@{}}$Z$\\initial productivity\end{tabular}
  & 1
  & \textcite{bilbiie2012endogenous} \\

\begin{tabular}[t]{@{}l@{}}$\chi$\\disutility of labor\end{tabular}
  & 0.924271
  & \textcite{bilbiie2012endogenous} \\

\begin{tabular}[t]{@{}l@{}}$\varphi$\\elasticity of labor supply\end{tabular}
  & 4
  & \textcite{bilbiie2012endogenous} \\

\begin{tabular}[t]{@{}l@{}}$\phi$\\elasticity in the entry function\end{tabular}
  & 0.3
  & \textcite{sedlavcek2017growth} \\

\begin{tabular}[t]{@{}l@{}}$\rho_X$\\Stock of Ideas shock persistence\end{tabular}
  & 0.415
  & \textcite{sedlavcek2017growth} \\

\begin{tabular}[t]{@{}l@{}}$\sigma_X$\\Stock of Ideas shock standard deviation\end{tabular}
  & 0.000009
  & \textcite{sedlavcek2017growth} \\

\begin{tabular}[t]{@{}l@{}}$\rho_Z$\\productivity shock persistence\end{tabular}
  & 0.9
  & \textcite{bilbiie2012endogenous} \\

\begin{tabular}[t]{@{}l@{}}$\sigma_Z$\\productivity shock standard deviation\end{tabular}
  & $\log (1.01)^2$
  & \textcite{bilbiie2012endogenous} \\

\midrule
\begin{tabular}[t]{@{}l@{}}$\alpha_i$\\returns to scale parameters for each of the 9 sectors\end{tabular}
  & [0.890,\,0.932,\,0.946,\,0.956,\,0.963,\,0.968,\,0.972,\,0.976,\,0.988]
  & average size in BDS  \\

\begin{tabular}[t]{@{}l@{}}$\Pi_i = \bigl(\psi_i / e_i\bigr)^{1-\varphi}$\\probability of successful entry for each of the 9 sectors\end{tabular}
  & [0.625,\,0.357,\,0.218,\,0.123,\,0.070,\,0.040,\,0.022,\,0.013,\,0.002]
  & firm shares in BDS  \\
\bottomrule
\end{tabular}
\end{sidewaystable}

For the C.E.S. preferences, I use the value of $\theta = 3.8$ from \textcite{bernard2003plants} 
adopted by BGM. I set $\beta = 0.96$ to match an annual real interest rate of 4\%. I set productivity $Z$ and 
the fixed entry cost $f_E$ both to $1$. Finally, I consider $\varphi = 0$ for the inelastic labor case, 
and $\varphi = 4$ for the elastic labor case, and I take $\chi = 0.924271$ from BGM directly for ease of comparison
with their baseline results. For the across-sector C.E.S. parameter, I take the calibration value of $\eta = 2$ from
\textcite{carvalho2021sectoral}, due to the similar sectoral structure found in their model.

For the Sedláček-Sterk matching block, I set the elasticity in the entry function to $\phi = 0.3$ to match their 
calibration. Following their approach with the BDS data, I set the total number of technology types equal to the
number of size groups available in the database, yielding $I = 9$ sectors/technology types. I also adopted their
implied values for the returns-to-scale parameters and matching probabilities, which they back out directly from
BDS data. 

Finally, for the two shocks adopted here, I take the exact same persistence and standard deviations used in the 
original BGM and Sedláček-Sterk exercises. The productivity shock has
persistence $\rho_Z =0.9$ and standard deviation $\sigma_Z = \log (1.01)^2$, whereas the stock-of-ideas shock has 
persistence $\rho_X = 0.415$ and variance $\sigma_X = 0.000009$. 

%%%%%%%%%%%%%%%%%%%%%%%%%%%%%%%%%%%%%%%%%%%%%%%%%%%%%%%%%%%%%%%%%%%%%%%%%%%%%
\subsection{Impulse responses}
\label{sec:quant-IRF}

\medskip
\medskip
\noindent\textbf{Productivity Shock}
\medskip

In Figure \ref{fig:multiprod} and Figure \ref{fig:aggprod}, the responses for the inelastic labor case
to a transitory one percent productivity shock are displayed for both sectoral variables in 
\ref{fig:multiprod} and aggregate ones in \ref{fig:aggprod}. As described before, periods are interpreted as years,
and the number of periods after the shock is on the horizontal axis. I start by analyzing the sectoral responses and
then move on to the aggregate ones. 

\begin{figure}[H]
  \centering
  \includegraphics[width=1\textwidth]{../images/inelastic_labor/Zshocks/irf_epsZ_sectoral.pdf}
  \caption{Responses to a transitory positive productivity shock for main sectoral variables.}
  \label{fig:multiprod}
\end{figure}

In Figure \ref{fig:multiprod}, the response of sectoral flow variables such as $M_i$, $e_i$, and $L_i$ are kept in level
deviation from steady-state, to make the absolute quantities more clear for sectoral comparison. The variables are not normalized, however,
so absolute values should not be interpreted literally. On the other hand, all other
variables are logged, such that the response represents percentage deviations from steady state, to characterize the heterogeneous
responses of different sectors. The only exception is the IRF for the real/relative price $\rho_i$, which is 
kept in levels to help clearly distinguish which sectors have higher and lower levels.

Initially, a temporary increase in TFP leads to a more attractive business environment, thus enticing
entry in all sectors. In particular, sectors with higher $\alpha$ parameters face more entry because of higher MPLs,
meaning that a rise in TFP yields higher returns. Due to the time-to-build lag, $M_i$ rises subsequently, but because
of the matching frictions, the sectoral differences are more balanced. Consequently, the relative price rises, in particular in
sectors that face more successful entry, due to the love-for-variety expression \eqref{eq:lovevariety}. Finally, a spike in
labor used in building new product lines can be seen, especially in high-$\alpha$ sectors since prospective entry is higher, demanding more
hours.

The most interesting dynamics, however, are the responses of the firm-level variables ($y_i,d_i,v_i,l_i$). In particular, the
negative dip that profits, output, and value suffer in high-$\alpha$ sectors. Initially, the temporary positive 
increase in $Z$ leads to bigger future profits and therefore bigger net present value of entering across all sectors\footnote{The productivity shock affects
all sectors with the same initial magnitude, and therefore, the deviation from steady state is the same across sector for firm-level variables.}, as can be
seen in Figure \ref{fig:multiprod}. Due to time-to-build lag, firms pay the entry cost today but only start producing
next period. Once this new cohort starts producing, the over-entry that happened, in particular in highly scalable sectors, leads
to a competition glut (excessively high number of incumbents $M_i$). 

This has two effects on firm-level variables. The first is the
increase in the relative price $\rho_i$, which leads to consumers substituting away from the sectors with the highest prices, decreasing
sectoral demand $C_i$, which forces firm-level output $y_i$ downwards. The second is the dilution of
firm-level output and profits due to the conditional symmetric equilibrium conditions, where more firms now share the same demand.
The higher the increase in $M_i$, the higher the impact of this mechanism, leading to a bigger impact on high-$\alpha$ sectors, as visible
in the figure. This dynamic is well-portrayed in the firm-level labor demand $l_i$, which decreases substantially since firms now produce 
less output.

The impact of the competition glut alongside the increase in relative prices is what ultimately leads to the negative dip seen in some firm-level variables in
Figure \ref{fig:multiprod}, in particular for sector 9, which, as explored before, represents a significant outlier 
and thus suffers from the competition glut the hardest. The recovery back to steady state happens as the exogenous firm exit shock
thins out the bloated number of incumbents $M_i$. This, in turn, leads to less competition and lowers relative prices, bringing $y_i$ and $d_i$
back and therefore also $v_i$. Low-$\alpha$ sectors never attract enough entrants to flip their incumbents into the red, so they just glide
back as Z dissipates.

Finally, the response of $M_i$ is also worth some exploration. $M_i$ is the key endogenous state in the model and,
much like in BGM, the sunk cost dynamic and time-to-build lag makes $M_i$ behave very much like capital in the 
baseline RBC. Thus, $e_i$ represents consumer investment, and $v_i$ is the key price for household financing, with $e_i$ initially overshooting
because $v_t$ is expected to temporarily increase in the future, especially in high $\alpha$ sectors. This makes it profitable for households
to invest today to reap the temporary benefits of the positive productivity shock.

Moving on to Figure \ref{fig:aggprod}, the aggregate responses to the productivity shock are now of interest. The response of labor used in
firm production ($Le$) is in line with the dynamics explored before, fuelling the substantial entry of new incumbents into the many sectors.
Labor used in production ($Lc$) then rises to match the production of these new firms after they are built, hence the lag. Real wage rises to
match the increase in labor demand for both entry and production.

\begin{figure}[H]
  \centering
  \includegraphics[width=0.7\textwidth]{../images/inelastic_labor/Zshocks/irf_epsZAgg.pdf}
  \caption{Responses to a transitory positive productivity shock for main aggregate variables.}
  \label{fig:aggprod}
\end{figure}

The most interesting effect, however, consists of the response of aggregate
output $Y$. In the baseline BGM framework, output responds to the one-percent increase in $Z$ with a slightly bigger response of 
around one to two percent. In Figure \ref{fig:aggprod}, however, the response far exceeds the one of BGM, leading to an increase of
around eleven percent. This happens through the love-for-variety effect found in BGM, which drives an increase
in output due to the increase in the number of varieties, but also due to two novel channels. The first is the possibility of cross-sector
C.E.S. substitution, where households redirect demand towards sectors whose relative price rise the less. Because the elasticity parameter
$\eta$ is bigger than one, the quantity response is elastic, such that the demand shift itself raises $Y$. This channel is sensitive to
the calibration of $\eta$ and thus its effect should be taken with a grain of salt.

The second novel channel constitutes the labor reallocation between sectors after a productivity shock. Since the effect of an increase in
TFP on the MPL depends directly on the returns-to-scale parameter $\alpha$, the shock induces 
reallocation from low-$\alpha$ to high-$\alpha$ sectors. This, in turn, raises output, since labor is now used more efficiently across 
the economy both for entry and production. The two novel channels represent extensions to the intensive composition margin in the model, 
going beyond the love-for-variety transmission mechanism found in BGM. Thus, the introduction
of heterogenous firm sectors into the BGM framework not only enriches the pre-existing dynamics but introduces new transmission mechanisms
that enrich the propagation and effect of productivity shocks in the model.

As found in the original BGM quantitative exercise with productivity shocks, the response in the elastic labor case is 
qualitatively similar for almost all variables. Figures \ref{fig:multiprod-elastic} and \ref{fig:aggprod-elastic} in the appendix present the IRFs for both
sectoral and aggregate variables, respectively. All sectoral variables follow the exact same pattern as in the inelastic labor case, albeit 
with stronger responses. That is because the household now has an additional margin of adjustment in the face of shocks, thus enhancing the model's 
propagation mechanism and amplifying the responses of most endogenous variables, since households adjust their hours worked. 

As for the aggregate
responses in Figure \ref{fig:aggprod-elastic}, all variables stay the same excluding labor that goes into production ($Lc$). For this variable, there
is an initial decline to levels lower than steady state, followed by a recovery and a brief period of above steady-state levels of work hours. 
The initial drop is not present in the inelastic labor case because now hours are free to adjust, so the income effect of the positive 
productivity shock makes households enjoy more leisure now, leading to a fall in production labor even though the marginal product of 
labor is higher. The subsequent rise comes from  the newly built product lines demanding labor to produce goods, as in the 
inelastic labor scenario.



\medskip
\medskip
\noindent\textbf{Stock-of-Ideas Shock}
\medskip

In Figure \ref{fig:multistock} and Figure \ref{fig:aggstock}, the responses for the inelastic labor case
to a transitory stock-of-ideas shock is displayed for both sectoral variables in 
\ref{fig:multistock} and aggregate ones in \ref{fig:aggstock}. As described before, periods are interpreted as years,
and the number of periods after the shock is on the horizontal axis. I start by analyzing the sectoral responses and
then move on to the aggregate ones. The stock-of-ideas shock follows the same magnitude as the composition shock found in
\textcite{sedlavcek2017growth}, that is, the shock is persistent ($\rho_X = 0.415$) but minuscule, with annual
innovation volatility of $\sigma_X = 0.000009$ (less than one-hundredth of a percent). This explains the minimal responses of
most endogenous variables, especially when compared to the responses to the productivity shock explored above.

As briefly discussed before in section \ref{sec:model-mine}, I choose to modify the original composition shock to have it affect the
stock of ideas of all sectors equally. That is because my goal is the study of aggregate RBC dynamics rather than cohort-by-year composition.
This makes sure that the shock now has a larger macro impact, functioning as a facilitator of business activity across the board, from high- 
to low-scalability types. Empirical analogues could be a patent-law change, the arrival of an innovation, or a broad surge in entrepreneurial 
funding. The bottom line is that the shock now endogenously models an aggregate event that facilitates starting businesses by affecting
the actual aggregate stock of ideas $\Psi$, preserving the microfoundations of the model. This differs from the literature that models entry-rate
shocks as exogenous "animal-spirits" phenomena (\cite{leduc2016uncertainty}), simply hardcoding a higher $e_i$.

\begin{figure}[H]
  \centering
  \includegraphics[width=1\textwidth]{../images/inelastic_labor/Xshocks/irf_epsX_sectoral.pdf}
  \caption{Responses to a transitory positive stock-of-ideas shock for main sectoral variables.}
  \label{fig:multistock}
\end{figure}

Despite the shock multiplying the stock of ideas for each sector equally, entry still differs due to different returns-to-scale parameters
and matching probabilities, as visible in Figure \ref{fig:multistock}. Prospective entry in all nine sectors jumps on impact due to a higher
number of ideas being available, a dynamic similar to an increase in the number of lottery tickets for a given prize (successful entry in
this scenario). As before, sector nine sees the highest surge in entrants by far, given the attractiveness of its high-$\alpha$ technology.
Due to time-to-build and sunk entry costs, the number of product lines $M_i$ in each sector adapts with a lag to the new entrants, behaving
again as physical capital in the textbook RBC model. A stark increase in sectoral-level labor $L$ to build new firms is once again present,
matching the spike in prospective entrants.

Echoing the previous results for the productivity shocks, the firm-level variable dynamics are plagued by the same curse of over-entry
and competition, but without the positive effect of a temporary rise in productivity to compensate. Therefore, due to the over-entry mechanism
and the subsequent competition glut explored before, firm-level output ($y_i$), labor ($l_i$), and profits ($d_i$) all fall with a lag after the
time-to-build lag. Evidently, high-$\alpha$ sectors suffer the biggest impact, with over-entry biting their profits the hardest. Firm value,
however, exhibits a different behavior than the three other variables. It instantly falls when the shock happens, since investors take the future
increase in competitors and thus lower profits into consideration. Interestingly, however, $v_i$ overshoots above steady-state levels for low-$\alpha$
sectors only, since these sectors face lesser future competition and therefore have the fastest profit recovery of all sectors. Ultimately, it is
the combination of the initial expected profit impact of entry and the speed of the expected relief that dictates the firm valuation dynamics.

\begin{figure}[H]
  \centering
  \includegraphics[width=0.7\textwidth]{../images/inelastic_labor/Xshocks/irf_epsX_Agg.pdf}
  \caption{Responses to a transitory positive stock-of-ideas shock for main aggregate variables.}
  \label{fig:aggstock}
\end{figure}

Moving on to aggregate variables in Figure \ref{fig:aggstock}, a similar dynamic as in Figure \ref{fig:aggprod} can be seen for both 
production and entry labor. Labor used in entry ($Le$) spikes initially to support the new entrants, with the labor used in production ($Lc$)
following suit to support production after the time-to-build lag. Output rises due to a sheer increase in the total number of producers, which,
due to the love-for-variety effect, fosters higher demand. Some reallocation of demand across sectors also happens, since the shock also causes
changes in the relative price $\rho_i$, but less than the response for the productivity shock. Finally, the real wage now rises with a lag since
labor demand increases in production only after firms are built, and the stock-of-ideas shock does not directly influence MPL on inpact and thus the 
real wagedoes not immediately rise as with the TFP shock.

Once more, the responses for the elastic labor case can be found in Figures \ref{fig:multistock-elastic} and \ref{fig:aggstock-elastic} 
in the appendix.
For the aggregate responses, a similar dynamic to the one found in Figure \ref{fig:aggprod-elastic} can be seen, with labor used in 
production ($Lc$) facing the initial dip as before, due to households now adjusting their labor supply in response to the income effect 
from higher labor demand in entry ($Le$).

On the other hand, sectoral-level responses for firm-level variables present a new dynamic not seen in the inelastic case. Namely, the dip in the
firm-level variables of employment ($l_i$), output ($y_i$), profits/dividends ($d_i$), and valuation ($v_i$) happens immediately after the shock,
without lagging. This is also due to households now being able to adjust their total labor supply. Now, households can use fewer hours in production
and more in entry since labor demand there is higher, as can be seen by the significantly higher levels of entry in Figure \ref{fig:multistock-elastic}
when compared to the inelastic case. Naturally, this leads to lower output, profits, and overall firm value, in practice bringing the 
over-entry/competition glut effect forward in time. The same effect is absent from the response in Figure \ref{fig:multiprod-elastic}, simply because
the productivity shock also raises the MPL in production, forcing households to keep their hours used in producing goods. 



%%%%%%%%%%%%%%%%%%%%%%%%%%%%%%%%%%%%%%%%%%%%%%%%%%%%%%%%%%%%%%%%%%%%%%%%%%%%%
\section{Conclusion}
\label{sec:conclusion}

In this thesis, I seek to understand how sectoral firm heterogeneity in returns to scale and entry frictions alter RBC shock propagation, and how
entry dynamics affect sectors differently.
To that intent, I embed the Sedláček-Sterk technology choice and matching into the Bilbiie-Ghironi-Melitz RBC with endogenous entry framework and explore
responses to both a TFP and stock-of-ideas shock.
Doing this amplifies the transmission of an aggregate positive TFP shock roughly five-fold and uncovers a competition-induced negative 
impact on firm-level profits of an increase in entry, which greatly varies by sector. 

The key findings can be separated into aggregate results, firm-level/sectoral results, and mechanistic insights. As for the aggregate results,
the main finding is the new magnitude of the response of aggregate output to the temporary TFP shock. Output jumps up around 11\%  (1-2\% in BGM) due to 
the love-for-variety effect from BGM, and two new channels: cross-sector C.E.S. substitution and labor reallocation toward high-$\alpha$ sectors.
The stock-of-ideas shock also drives a modest increase in GDP due to an increase in varieties, but real wages lag until firms start producing.

Moving on to firm-level/sectoral results, the findings suggest that over-entry and competition glut push profits, output, and employment per firm 
temporarily below steady state after a positive TFP or stock-of-ideas shock, with the effect being the strongest in high-$\alpha$ sectors, due to higher
expected profits. Sectoral heterogeneity also influences steady-state properties, where highly scalable sectors showcase higher labor demand and output 
and overall number of varieties than their less scalable counterparts.

Finally, the findings echo the original BGM interpretation that entry behaves like capital investment in a classic RBC, such that the total number of
incumbents evolves with time-to-build and depreciation-like exit, strengthening persistence. On top of that, however, the expanded model allows 
different investment responses for different sectors and adds a new potential source of investment loss through the matching frictions, 
which further bolsters persistence in the model. Steady-state entry dynamics also show decreasing "returns-to-entry", where each successive sector is 
fighting steeper diminishing chances of success than the previous one, raising losses at entry.

Policy interventions should therefore be more granular in their incentives. For instance, startup support that varies by scalability class helps avoid 
the blanket subsidy pitfall, where incentives can exacerbate the competition glut identified in highly scalable (high-$\alpha$) sectors. 
Steering some of those incentives toward less-scalable industries would spread entrepreneurial effort more evenly, easing matching 
frictions and the losses they create. Downturn subsidies should also factor in heterogeneous variety externalities and sectoral misallocation, 
nudging entrants to internalize the consumer surplus gains that new varieties generate based on their sector.

Some potential caveats qualify the main conclusions, however. First, calibration relies heavily on BDS-implied data and the sectoral splits made by 
\textcite{sedlavcek2017growth}, such that results could be sensitive to alternative splits or new data altogether. Second, constant markups mute 
potential endogenous markup/price dispersion channels that naturally play an important role in firm dynamics. Third, all firms, no matter their scalability,
face the same uniform exogenous exit probability, biasing competitive pressure and persistence. Fourth, the lack of capital shuts down capital-adjustment
dynamics that usually amplify RBC persistence. Finally, ignoring household heterogeneity and portfolio risk may understate the distributional
effects, particularly when entry reallocates labor across sectors.

Therefore, there are many potential avenues for future research. The first and most obvious one would be experimenting with endogenous markups generated by
translog preferences, and implementing capital into the model. Both expansions are done in BGM, where the authors find a significant
impact of translog preferences through endogenous markups on shock transmission, and that the addition of capital improves RBC second-moments significantly.
Another potential idea is exploring more in depth the welfare impacts of variety externalities in a sectoral firm heterogeneity setup. As discussed before, 
this could yield interesting and novel policy insights not found in the baseline BGM framework. Additionally, comparing the framework with frameworks 
that have different sources of heterogeneity, such as markups or productivity parameters, could also be fruitful. 

Overall, recognizing firm differences in scalability and the resulting matching frictions enhances how 
shock transmission is understood in RBC models and can guide smarter policy entry during downturns that respects
the heterogeneous responses of firms. 



%%%%%%%%%%%%%%%%%%%%%%%%%%%%%%%%%%%%%%%%%%%%%%%%%%%%%%%%%%%%%%%%%%%%%%%%%%%%%
%%%%%%%%%%%%%%%%%%%%%%%%%%%%%%%%%%%%%%%%%%%%%%%%%%%%%%%%%%%%%%%%%%%%%%%%%%%%%
%%%%%%%%%%%%%%%%%%%%%%%%%%%%%%%%%%%%%%%%%%%%%%%%%%%%%%%%%%%%%%%%%%%%%%%%%%%%%
% References section
\newpage
\thispagestyle{plain}
\pagenumbering{Roman}
\printbibliography[heading=bibintoc] % Insert references

%%%%%%%%%%%%%%%%%%%%%%%%%%%%%%%%%%%%%%%%%%%%%%%%%%%%%%%%%%%%%%%%%%%%%%%%%%%%%
%%%%%%%%%%%%%%%%%%%%%%%%%%%%%%%%%%%%%%%%%%%%%%%%%%%%%%%%%%%%%%%%%%%%%%%%%%%%%
%%%%%%%%%%%%%%%%%%%%%%%%%%%%%%%%%%%%%%%%%%%%%%%%%%%%%%%%%%%%%%%%%%%%%%%%%%%%%
% Appendices section
\newpage
\begin{refsection}
\thispagestyle{plain}
\pagenumbering{arabic}  % Turn page numbering to arabic
\renewcommand*{\thepage}{A-\arabic{page}} % Add 'A' to each page number for appendices section
\addtocontents{toc}{\protect\setcounter{tocdepth}{1}} % This hides the appendix subsections in the table of contents
\begin{appendices}
%%%%%%%%%%%%%%%%%%%%%%%%%%%%%%%%%%%%%%%%%%%%%%%%%%%%%%%%%%%%%%%%%%%%%%%%%%%%%
%%%%%%%%%%%%%%%%%%%%%%%%%%%%%%%%%%%%%%%%%%%%%%%%%%%%%%%%%%%%%%%%%%%%%%%%%%%%%
\section{Additional Figures}
\label{sec-app:full}

\begin{figure}[H]
  \centering
  \includegraphics[width=1\textwidth]{../images/elastic_labor/Zshocks/irf_epsZ_sectoral.pdf}
  \caption{Responses to a transitory positive productivity shock for main sectoral variables (Elastic Labor).}
  \label{fig:multiprod-elastic}
\end{figure}

\begin{figure}[H]
  \centering
  \includegraphics[width=0.7\textwidth]{../images/elastic_labor/Zshocks/irf_epsZAgg.pdf}
  \caption{Responses to a transitory positive productivity shock for main aggregate variables (Elastic Labor).}
  \label{fig:aggprod-elastic}
\end{figure}

\begin{figure}[H]
  \centering
  \includegraphics[width=1\textwidth]{../images/elastic_labor/Xshocks/irf_epsX_sectoral.pdf}
  \caption{Responses to a transitory positive stock-of-ideas shock for main sectoral variables (Elastic Labor).}
  \label{fig:multistock-elastic}
\end{figure}

\begin{figure}[H]
  \centering
  \includegraphics[width=0.7\textwidth]{../images/elastic_labor/Xshocks/irf_epsX_Agg.pdf}
  \caption{Responses to a transitory positive stock-of-ideas shock for main aggregate variables (Elastic Labor).}
  \label{fig:aggstock-elastic}
\end{figure}

\pagebreak
%%%%%%%%%%%%%%%%%%%%%%%%%%%%%%%%%%%%%%%%%%%%%%%%%%%%%%%%%%%%%%%%%%%%%%%%%%%%%
\section{Computational Implementation}
\label{sec-app:codes}

The computational appendix provides a guide to the code repository accompanying this thesis. All scripts and data files are available at the GitHub repository:
\begin{center}
  \url{https://github.com/Okim343/MyHeteroBilbiie_Thesis.git}
\end{center}

\subsection{Setup}

The code is written in Julia (v1.9.4) and uses the \texttt{Dynare.jl} package for solving and simulating the 
RBC model for both the inelastic and elastic labor specifications. To install the correct Julia version and dependencies:

\begin{enumerate}
  \item Install Julia 1.9.4 via \texttt{juliaup}.
  \item Open a Julia REPL in the project root and run:
    \begin{verbatim}
    ] activate .
    (MyHeteroBilbiie) pkg> instantiate
    \end{verbatim}
\end{enumerate}

\subsection{Folder Structure}

\begin{itemize}
  \item \texttt{src/}:  
    Contains the main Julia scripts that solve the model, compute impulse responses, and generate all figures.
  \item \texttt{helper\_functions/}:  
    Utility routines for computing steady states (\texttt{steady\_state.jl}), drawing figures (\texttt{steady\_state\_figures.jl}), and plotting IRFs (\texttt{plot\_irfs.jl}).
  \item \texttt{DynareModFiles/}:  
    Dynare model files (\texttt{calibration.mod}, \texttt{declarations.mod}, \texttt{model.mod}, etc.) for both elastic and inelastic labor specifications.
  \item \texttt{Thesis/}:  
    The \LaTeX{} source of the thesis, including the compiled PDF and bibliography.
\end{itemize}

\subsection{Scripts}

To reproduce the results presented in the thesis, execute the following in order:

\begin{description}
  \item[\texttt{analyze\_steadystate.jl}]: 
    Solves and visualizes the steady state of the heterogeneous‐firm RBC model under both elastic and inelastic labor. Prints steady‐state values for all key variables.
  \item[\texttt{run\_model.jl}]: 
    Loads the calibrated Dynare model with elastic labor, runs perfect‐foresight and stochastic simulations, computes impulse response functions for aggregate shocks, and exports all sectoral and aggregate IRF plots.
  \item[\texttt{run\_model\_ineL.jl}]: 
    Analogous to \texttt{run\_model.jl}, but for the inelastic labor specification.
\end{description}

\subsection{Reproducing Figures}

After obtaining steady‐state values from \texttt{analyze\_steadystate.jl}, these must be copied into the Dynare mod files (\texttt{calibration.mod} and \texttt{calibration\_ineL.mod}) because \texttt{Dynare.jl} does not yet support dynamic parameter passing. Then:

\begin{enumerate}
  \item Run \texttt{run\_model.jl} to generate IRFs for the elastic labor case.
  \item Run \texttt{run\_model\_ineL.jl} to generate IRFs for the inelastic labor case.
  \item The resulting figures are saved under \texttt{Images/}, organized by specification 
  (elastic/inelastic) and shock type (XShock, ZShock).
\end{enumerate}


\pagebreak
%%%%%%%%%%%%%%%%%%%%%%%%%%%%%%%%%%%%%%%%%%%%%%%%%%%%%%%%%%%%%%%%%%%%%%%%%%%%%
\section{No Option Value for Waiting to Enter with Matching Frictions}
\label{sec-app:option}

The following proof is very similar to the one found in BGM, with the only major difference being the addition of $\Pi_{i,t}$,
the mathing probability at time $t$ for sector $i$, and that firms are now characterized by both $i$ and $\omega$.
Let the option value of waiting to enter for firm $\omega i$ be $W_{i,t}(\omega)\ge0$. In all periods $t$,
\[
W_{i,t}(\omega)
=\max\!\Bigl[v_t(\omega)-\frac{w_t\,f_{E,t}}{\Pi_{i,t}Z_t},\;\beta\,W_{i,t+1}(\omega)\Bigr],
\]
where the first term is the payoff of undertaking the investment and the second term is the discounted payoff of waiting. 
If firms are identical (there is no idiosyncratic uncertainty) and exit is exogenous (uncertainty related to firm death is also aggregate), this becomes:
\[
W_{i,t}
=\max\!\Bigl[v_t-\frac{w_t\,f_{E,t}}{\Pi_{i,t}Z_t},\;\beta\,W_{i,t+1}\Bigr].
\]
Because of free entry, the first term is always zero, so the option value obeys:
\[
W_{i,t} = \beta\,W_{i,t+1}.
\]
This is a contraction mapping because of discounting, and by forward iteration, under the assumption
\[
\lim_{T\to\infty}\beta^T\,W_{i,t+T}=0
\]
(i.e., there is a zero value of waiting when reaching the terminal period), the only stable solution for the option value is
\[
W_{i,t} = 0 \quad \forall i \in \{1,2,...,I\}.
\]

\pagebreak
%%%%%%%%%%%%%%%%%%%%%%%%%%%%%%%%%%%%%%%%%%%%%%%%%%%%%%%%%%%%%%%%%%%%%%%%%%%%%
\section{Supplementary Results}
\label{sec-app:sup}

This appendix contains supplementary figures and tables, placed here for the sake of brevity in the main text. It is ordered according to the corresponding sections in the text.

\begin{table}[htbp]
\centering
\label{tab:moments}
% -------- first row: σ  and  relative σ ------------------------------------
\begin{subtable}[t]{.48\textwidth}
  \centering
  \caption*{(a) Volatility $\sigma_X$}
  \begin{tabular}{lcccc}
  \toprule
   $X$ & \dat{Data} & Bilbiie & \rbc{RBC} & \mine{Mine}\\
  \midrule
   $Y_R$      & \dat{1.81} & 1.34 & \rbc{1.39} & \mine{0.99}\\
   $C_R$      & \dat{1.35} & 0.65 & \rbc{0.61} & \mine{0.83}\\
   $v_RN_E$   & \dat{5.30} & 5.23 & \rbc{4.09} & \mine{1.57}\\
   $L$        & \dat{1.79} & 0.63 & \rbc{0.67} & \mine{0.25}\\
  \bottomrule
  \end{tabular}
\end{subtable}
\hfill
\begin{subtable}[t]{.48\textwidth}
  \centering
  \caption*{(b) Relative volatility $\sigma_X/\sigma_{Y_R}$}
  \begin{tabular}{lcccc}
  \toprule
   $X$ & \dat{Data} & Bilbiie & \rbc{RBC} & \mine{Mine}\\
  \midrule
   $Y_R$      & \dat{1.00} & 1.00 & \rbc{1.00} & \mine{1.00}\\
   $C_R$      & \dat{0.74} & 0.48 & \rbc{0.44} & \mine{0.84}\\
   $v_RN_E$   & \dat{2.93} & 3.90 & \rbc{2.95} & \mine{1.59}\\
   $L$        & \dat{0.99} & 0.47 & \rbc{0.48} & \mine{0.25}\\
  \bottomrule
  \end{tabular}
\end{subtable}

\vspace{0.8em}   % space between the two rows of panels

% -------- second row: persistence  and  correlations -----------------------
\begin{subtable}[t]{.48\textwidth}
  \centering
  \caption*{(c) Persistence $E[X_t X_{t-1}]$}
  \begin{tabular}{lcccc}
  \toprule
   $X$ & \dat{Data} & Bilbiie & \rbc{RBC} & \mine{Mine}\\
  \midrule
   $Y_R$      & \dat{0.84} & 0.70 & \rbc{0.72} & \mine{0.73}\\
   $C_R$      & \dat{0.80} & 0.75 & \rbc{0.79} & \mine{0.74}\\
   $v_RN_E$   & \dat{0.87} & 0.69 & \rbc{0.71} & \mine{0.71}\\
   $L$        & \dat{0.88} & 0.69 & \rbc{0.71} & \mine{0.74}\\
  \bottomrule
  \end{tabular}
\end{subtable}
\hfill
\begin{subtable}[t]{.48\textwidth}
  \centering
  \caption*{(d) Contemporaneous corr.\ with $Y_R$}
  \begin{tabular}{lcccc}
  \toprule
   $X$ & \dat{Data} & Bilbiie & \rbc{RBC} & \mine{Mine}\\
  \midrule
   $Y_R$      & \dat{1.00} & 1.00 & \rbc{1.00} & \mine{1.00}\\
   $C_R$      & \dat{0.88} & 0.97 & \rbc{0.94} & \mine{0.99}\\
   $v_RN_E$   & \dat{0.80} & 0.99 & \rbc{0.99} & \mine{0.97}\\
   $L$        & \dat{0.88} & 0.98 & \rbc{0.97} & \mine{0.99}\\
  \bottomrule
  \end{tabular}
\end{subtable}
\caption{Second-moment statistics: \dat{Data}, Bilbiie et al.\ (2012), 
\rbc{Baseline RBC}, and \mine{My Model}}
\end{table}



\newpage
%%%%%%%%%%%%%%%%%%%%%%%%%%%%%%%%%%%%%%%%%%%%%%%%%%%%%%%%%%%%%%%%%%%%%%%%%%%%%
%%%%%%%%%%%%%%%%%%%%%%%%%%%%%%%%%%%%%%%%%%%%%%%%%%%%%%%%%%%%%%%%%%%%%%%%%%%%%
\end{appendices}
\thispagestyle{plain}
\pagenumbering{Roman} % Start Roman page numbering for appendix references
\renewcommand*{\thepage}{A-\Roman{page}} % Add 'A' to each page number for appendix references

\addcontentsline{toc}{section}{Appendix References}
\printbibliography[heading=subbibliography, title={Appendix References}]
\thispagestyle{plain}
\cleardoublepage % Ensure the next content starts on an odd page (if your document is two-sided)

\end{refsection}

%%%%%%%%%%%%%%%%%%%%%%%%%%%%%%%%%%%%%%%%%%%%%%%%%%%%%%%%%%%%%%%%%%%%%%%%%%%%%
%%%%%%%%%%%%%%%%%%%%%%%%%%%%%%%%%%%%%%%%%%%%%%%%%%%%%%%%%%%%%%%%%%%%%%%%%%%%%
%%%%%%%%%%%%%%%%%%%%%%%%%%%%%%%%%%%%%%%%%%%%%%%%%%%%%%%%%%%%%%%%%%%%%%%%%%%%%
\newpage
\thispagestyle{plain}
\pagenumbering{gobble} % Turn page numbering off
\section*{Statement of Authorship} % Include statement of authorship
I hereby confirm that the work presented has been performed and interpreted solely by myself except for where I explicitly identified the contrary. I assure that this work has not been presented in any other form for the fulfillment of any other degree or qualification. Ideas taken from other works in letter and in spirit are identified in every single case.

\vspace{2cm}
\noindent
\rule{8cm}{0.4pt}\\
Enrico Truzzi\\
Bonn, the \printdate{2025-7-1}
\end{document}
% End Document
%%%%%%%%%%%%%%%%%%%%%%%%%%%%%%%%%%%%%%%%%%%%%%%%%%%%%%%%%%%%%%%%%%%%%%%%%%%%%
%%%%%%%%%%%%%%%%%%%%%%%%%%%%%%%%%%%%%%%%%%%%%%%%%%%%%%%%%%%%%%%%%%%%%%%%%%%%%
%%%%%%%%%%%%%%%%%%%%%%%%%%%%%%%%%%%%%%%%%%%%%%%%%%%%%%%%%%%%%%%%%%%%%%%%%%%%%
%%%%%%%%%%%%%%%%%%%%%%%%%%%%%%%%%%%%%%%%%%%%%%%%%%%%%%%%%%%%%%%%%%%%%%%%%%%%%
%%%%%%%%%%%%%%%%%%%%%%%%%%%%%%%%%%%%%%%%%%%%%%%%%%%%%%%%%%%%%%%%%%%%%%%%%%%%%
%%%%%%%%%%%%%%%%%%%%%%%%%%%%%%%%%%%%%%%%%%%%%%%%%%%%%%%%%%%%%%%%%%%%%%%%%%%%%
